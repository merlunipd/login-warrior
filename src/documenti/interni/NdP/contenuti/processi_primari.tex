\chapter{Processi Primari}

\section{Fornitura}
\subsection{Descrizione}
Nella seguente sezione sono riportate tutte le regole che ogni membro del gruppo \textit{MERL} si impegna a rispettare per mantenere attivi e produttivi i rapporti con il proponente$_G$ \textit{Zucchetti S.p.A.}.

\subsection{Scopo}
Il processo di fornitura ha lo scopo di risolvere ogni dubbio legato alla volontà del proponente, inseguendo così la totale comprensione delle richieste al fine di soddisfarle a pieno.

\subsection{Aspettative}
Durante l'intera realizzazione del progetto il gruppo \textit{MERL} intende mantenere rapporti frequenti con l'azienda \textit{Zucchetti S.p.A.} al fine di evitare qualsiasi tipo di incomprensione dovuta alla mancanza di dialogo. Riteniamo di fondamentale importanza avere un riferimento in grado di darci supporto durante l'intero svolgimento. Di seguito i punti fondamentali a cui daremo primaria importanza in questo rapporto:
\begin{itemize}
 \item Determinare i bisogni che il prodotto finale deve soddisfare;
 \item Stabilire le tempistiche di consegna del prodotto;
 \item Ottenere feedback$_G$ riguardo al lavoro svolto;
 \item Ottenere chiarimenti relativi a possibili dubbi o incomprensioni;
 \item Definire i vari vincoli sia riguardanti il prodotto finale sia riguardanti i processi intermedi che verranno attuati.
\end{itemize}

\section{Sviluppo}

\subsection{Descrizione}
  Lo scopo del processo di sviluppo è quello di raggruppare compiti e attività relative alla codifica di un prodotto software, applicandole durante il suo intero ciclo di vita$_G$. \\
  Al fine di produrre un software che rispetti le aspettative del proponente è necessario:
  \begin{itemize}
    \item Determinare i vincoli tecnologici;
    \item Determinare gli obiettivi di sviluppo e design$_G$;
    \item Realizzare un prodotto software che superi tutti i test di verifica$_G$ e di validazione$_G$.
  \end{itemize}
  \subsection{Attività}

  Il processo di sviluppo prevede l'esecuzione delle seguenti attività:
  \begin{itemize}
    \item \textbf{Analisi dei requisiti};
    \item \textbf{Progettazione};
    \item \textbf{Codifica}.
  \end{itemize}
  \subsection{Analisi dei requisiti}
    L'analisi dei requisiti$_G$ è quell'attività che precede lo sviluppo e ha la funzione di:
    \begin{itemize}
      \item Definire lo scopo del prodotto da realizzare;
      \item Definire gli attori$_G$ del sistema;
      \item Fissare le funzionalità$_G$ del prodotto;
      \item Fornire una visione più chiara del problema ai progettisti;
      \item Fornire un riferimento ai verificatori per l'attività di controllo dei test;
      \item Fornire una stima della mole di lavoro.
    \end{itemize}
    \subsubsection{Scopo}
    Lo scopo dell'attività è quello di fornire in un documento tutti i requisiti individuati.\\
    Al fine di individuarli è necessario:
    \begin{itemize}
      \item Leggere e comprendere la specifica del capitolato;
      \item Mantenere un confronto costante con il proponente.
    \end{itemize}
    \subsubsection{Casi d'uso}
    Definiscono uno scenario in cui uno o più attori interagiscono con il sistema. Sono identificati nel modo seguente:
    \begin{center}
      \textbf{UC[Numero caso d'uso].[Sottocaso]-[Titolo caso d'uso]}\\
    \end{center}
    \subsubsection{Struttura dei requisiti}
      Il codice identificativo di ciascun requisito$_G$ è di seguito riportato:
      \begin{center}
        \textbf{R[Tipologia].[Importanza].[Codice]}\\
      \end{center}
      con:
      \begin{itemize}
        \item Tipologia:
        \begin{itemize}
          \item \textbf{F}: requisito funzionale (servizi e funzioni offerti dal sistema);
          \item \textbf{Q}: requisito di qualità (vincoli di qualità, vedere il \textit{Piano di Qualifica V2.0.0} per i dettagli);
          \item \textbf{V}: requisito di vincolo (vincoli sui servizi offerti dal sistema);
          \item \textbf{P}: requisito prestazionale (vincoli sulle prestazioni da soddisfare).
        \end{itemize}

        \item Importanza:
        \begin{itemize}
          \item \textbf{1} requisito obbligatorio;
          \item \textbf{2} requisito desiderabile ma non obbligatorio;
          \item \textbf{3} requisito opzionale.
        \end{itemize}

        \item Codice:
        \begin{itemize}
          \item Identificativo del requisito che risulta essere univoco in base alla tipologia. In alcuni casi il codice presenta dei sottocasi identificati a loro volta con un "." (punto) seguito dal corrispondente valore del sottocaso.
        \end{itemize}

      \end{itemize}

  \subsection{Progettazione}
  \subsubsection{Scopo}
  Lo scopo della progettazione è quello di definire una possibile soluzione ai requisiti evidenziati dall'analisi.
  \subsubsection{Descrizione}
  La progettazione è formata da due parti:
  \begin{itemize}
    \item \textbf{Progettazione logica:} motiva le tecnologie, i framework$_G$ e le librerie$_G$ usate per la realizzazione di un prodotto, dimostrandone l'adeguatezza nel PoC$_G$.
    \\Contiene:
    \begin{itemize}
      \item I framework e le tecnologie utilizzate;
      \item Il Proof of Concept$_G$;
      \item I diagrammi UML$_G$.
    \end{itemize}
    \item \textbf{Progettazione di dettaglio:} illustra la base architetturale del prodotto coerentemente a ciò che è previsto nella Progettazione logica.
    \\Contiene:
    \begin{itemize}
      \item Diagrammi delle classi;
      \item Tracciamento delle classi;
      \item Test di unità per ogni componente.
    \end{itemize}
  \end{itemize}

  \subsection{Codifica}
  \subsubsection{Scopo}
  Lo scopo della codifica è quello di implementare le specifiche individuate in un prodotto utilizzabile.
  \subsubsection{Commenti}
  Nel caso sorga la necessità di scrivere qualche commento al codice è preferibile che esso sia chiaro e conciso.
  \subsubsection{Nomi dei file}
  I nomi dei file devono:
  \begin{itemize}
    \item Essere univoci;
    \item Esplicitare il contenuto dei file stessi.
  \end{itemize}
