\chapter{Introduzione}

\section{Scopo del Documento}
Lo scopo del documento è quello di definire le regole che ogni membro del gruppo \textit{MERL} deve rispettare al fine di raggiungere in modo efficace ed efficiente la creazione del prodotto finale.

Vengono inoltre specificate le convenzioni sull'uso dei vari strumenti scelti per la realizzazione del prodotto e vengono illustrati i processi che saranno adottati dal gruppo.

\section{Scopo del Prodotto}
Al giorno d'oggi ogni servizio presente sul web richiede un'autenticazione tramite login, fase fondamentale per la protezione dei dati di un individuo. Risulta ancora più importante se viene considerata la possibile presenza di malintenzionati con lo scopo di rubare ciò che dovrebbe essere privato. La presenza di attacchi informatici negli anni è andata aumentando e continua tuttora a crescere, per questo è necessario che questa pratica venga il più possibile riconosciuta e arginata.

Il capitolato C5 ha proprio come obiettivo quello di trovare una soluzione a questo problema. L'idea è quella di riconoscere le attività lecite e quelle illecite attraverso la raccolta, l'analisi e la visualizzazione di dati sotto forma di grafici e modelli che permettano un riconoscimento immediato delle differenze nei tentativi di accesso.

Con questo scopo il gruppo \textit{MERL} si impegnerà nella realizzazione di un'applicazione web in grado di leggere grandi quantità di dati di login per poi mostrare tramite dei grafici la natura di questi, riuscendo nell'intento di riconoscere a primo impatto le attività sospette.

\section{Glossario}
Al fine di evitare incomprensioni relative alla terminologia usata all'interno del documento, viene inserito un \textit{Glossario} in grado di dare una definizione precisa per ogni vocabolo potenzialmente ambiguo.

\section{Riferimenti}
\subsection{Riferimenti normativi}
\begin{itemize}
  \item Capitolato d'appalto C5 - \textit{Zucchetti S.p.A.}: \textit{Login Warrior} \\
  \url{https://www.math.unipd.it/~tullio/IS-1/2021/Progetto/C5.pdf}.
\end{itemize}

\subsection{Riferimenti informativi}
\begin{itemize}
  \item Slide T3 - Corso di Ingegneria del Software - Processi di ciclo di vita \\
  \url{https://www.math.unipd.it/~tullio/IS-1/2021/Dispense/T03.pdf};
  \item Slide FC1 - Corso di Ingegneria del Software - Amministrazione di progetto \\
  \url{https://www.math.unipd.it/~tullio/IS-1/2020/Dispense/FC1.pdf}.
\end{itemize}
