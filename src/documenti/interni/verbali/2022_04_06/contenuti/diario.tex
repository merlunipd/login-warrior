\section{Diario della riunione}
\begin{itemize}
  \item Sprint 6 review:
        \begin{itemize}
          \item Obiettivi raggiunti:
                \begin{itemize}
                  \item Progettazione: stesa bozza dell'architettura;
                  \item Discusso con il proponente riguardo alla configurazione dei grafici e architettura del prodotto;
                  \item Aggiornato \textit{NdP} con il nuovo metodo a "sprint" (sprint rapidi di circa una settimana, con planning, review e retrospective);
                  \item Aggiornato \textit{PdQ} con valutazioni per il miglioramento e testing;
                  \item Ricercato visualizzazioni del Force-Directed Graph.
                \end{itemize}
          \item Obiettivi non raggiunti:
                \begin{itemize}
                  \item Progettazione: non conclusa (difficoltà a contattare il docente Cardin per rispondere ad alcuni dubbi).
                \end{itemize}
          \item Consuntivo risorse utilizzate.
        \end{itemize}
\end{itemize}


\begin{itemize}
  \item Sprint 6 retrospective:
        \begin{itemize}
          \item Consolidamento cose positive:
                \begin{itemize}
                  \item Ottima l'idea di dividere il team in due sotto-gruppi (di 3, 4 persone) con compiti indipendenti per permettere di lavorare in parallelo e con maggiore agilità e flessibilità.
                \end{itemize}
          \item Miglioramento cose negative:
                \begin{itemize}
                  \item Con il nuovo metodo di lavoro agile, con sprint molto brevi e team piccoli con obiettivi chiari, i diagrammi di Gantt iniziano ad avere un costo molto alto in termini di risorse e un guadagno molto basso.
                \end{itemize}
        \end{itemize}
\end{itemize}


\begin{itemize}
  \item Sprint 7 planning:
        \begin{itemize}
          \item Obiettivi individuati:
                \begin{itemize}
                  \item Progettazione: terminare (con feedback del proponente e del docente Cardin);
                  \item Codifica:
                        \begin{itemize}
                          \item Essere in grado di avere la funzione \texttt{draw()} per ogni tipologia di grafico;
                          \item Interazione con \textit{IndexedDB};
                          \item Trasposizione delle funzionalità del PoC sull'architettura.
                        \end{itemize}
                  \item Aggiornare documenti: NdP, PdQ;
                  \item Ricerca:
                        \begin{itemize}
                          \item Algoritmo di campionamento dei dati (oltre a quello casuale);
                          \item Test su viste e controller.
                        \end{itemize}
                \end{itemize}
          \item Divisione team:
                \begin{itemize}
                  \item Progettazione: Marco Mazzucato, Emanuele Pase, Marko Vukovic;
                  \item Grafici: Riccardo Contin, Marco Mamprin, Mattia Zanellato;
                  \item Documenti: Lorenzo Onelia.
                \end{itemize}
        \end{itemize}
\end{itemize}