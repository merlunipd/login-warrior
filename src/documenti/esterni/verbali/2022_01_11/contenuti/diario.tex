\section{Diario della riunione}
\begin{itemize}
  \item Discussione con il proponente sull'Analisi dei Requisiti svolta fino a questo momento. In particolare sono emersi i seguenti punti:
  \begin{itemize}
    \item Dubbi sull'utilizzo delle estensioni negli UML inseriti nell'analisi dei Casi d'Uso;
    \item Discussione sull'utilizzo di file CSV o JSON;
    \item Rivisitazione degli UML inseriti al fine di renderli più coerenti possibili con la loro spiegazione testuale;
    \item Chiarimento sulla differenza tra manuale utente e manuale sviluppatore.
  \end{itemize}
  \item Risposta del proponente ad alcune domande:

  \begin{spacing}{2}
  \end{spacing}

  \begin{minipage}[b]{0.47\textwidth}
  \centering
  \textbf{Domande}
  \end{minipage}
  \hfill
  \begin{minipage}[b]{0.47\textwidth}
  \centering
  \textbf{Risposte}
  \end{minipage}

  \begin{spacing}{3}
  \end{spacing}

  \begin{minipage}[c]{0.47\textwidth}
  \centering
  Informazioni o particolari richieste sul PoC.
  \end{minipage}
  \hfill
  \begin{minipage}[c]{0.47\textwidth}
  \centering
  Sostanzialmente il PoC servirà a far emergere tutte le difficoltà di cui ancora non sappiamo l'esistenza e per questo motivo presumibilmente nel PoC verranno caricati molti meno dati rispetto al prodotto finale.
  \end{minipage}

  \begin{spacing}{3}
  \end{spacing}

  \begin{minipage}[c]{0.47\textwidth}
  \centering
  Risulta necessario sviluppare un login?
  \end{minipage}
  \hfill
  \begin{minipage}[c]{0.47\textwidth}
  \centering
  La richiesta del capitolato è incentrata sulla visualizzazione dei dati ed è su questo che ci viene chiesto di concentrare il lavoro, di conseguenza non è necessario sviluppare una procedura di login.
  \end{minipage}

  \begin{spacing}{3}
  \end{spacing}

  \begin{minipage}[c]{0.47\textwidth}
  \centering
  In termini di file e server, come verrà consegnato il prodotto finale?
  \end{minipage}
  \hfill
  \begin{minipage}[c]{0.47\textwidth}
  \centering
  Il prodotto finale sarà una single-page application e di conseguenza potranno anche essere consegnati solo i file necessari. Per quanto riguarda il discorso server è irrilevante al momento, nel senso che se il prodotto funziona caricarlo su un server risulta semplice.
  \end{minipage}

  \begin{spacing}{3}
  \end{spacing}

  \begin{minipage}[c]{0.47\textwidth}
  \centering
  Per quanto riguarda le tecnologie, ci è stato consigliato in precedenza l'utilizzo di \textit{D3.js}, ci sono altri consigli riguardo altre tecnologie?
  \end{minipage}
  \hfill
  \begin{minipage}[c]{0.47\textwidth}
  \centering
  La libreria \textit{D3.js} è stata consigliata perché al momento è probabilmente la migliore per il lavoro che dobbiamo svolgere, ma questo non è un vincolo obbligatorio quindi non toglie la possibilità di usare anche altre tecnologie in caso risultino utili e favorevoli.
  \end{minipage}

  \begin{spacing}{3}
  \end{spacing}

  \begin{minipage}[c]{0.47\textwidth}
  \centering
  È possibile effettuare unit testing sulla libreria \textit{D3.js} o in generale sulla parte grafica di visualizzazione?
  \end{minipage}
  \hfill
  \begin{minipage}[c]{0.47\textwidth}
  \centering
  Sostanzialmente risulta difficile fare unit testing su una visualizzazione grafica, tuttavia può essere utile il tool \textbf{Selenium} (\url{https://www.selenium.dev/}) che risulta essere ciò che più si avvicina allo unit testing su una visualizzazione grafica.
  \end{minipage}

\end{itemize}
