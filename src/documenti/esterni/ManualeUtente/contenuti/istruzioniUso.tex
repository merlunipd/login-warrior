\chapter{Istruzioni all'uso}
Il seguente capitolo fornirà tutte le spiegazioni per il corretto utilizzo del prodotto.

\section{Home}
Questa è la prima schermata disponibile dell'applicazione. Qui è possibile caricare i dati sotto forma di dataset o di sessione salvata in precedenza.

\begin{figure}[H]
    \includegraphics[width=1.0\textwidth]{Home.jpg}
    \caption{Screenshot della Home}
\end{figure}

\subsection{Carica dataset}
Cliccando su "Carica dataset" comparirà una finestra di dialogo che ci permetterà di caricare il dataset. Seleziona quindi il file in formato .csv desiderato e poi clicca su "Apri" per caricare il file.

\begin{figure}[H]
    \includegraphics[width=1.0\textwidth]{BottoneDataset.jpg}
    \caption{Screenshot della finestra di dialogo per il caricamento del dataset}
\end{figure}

Dopo aver caricato il dataset comparirà una lista dei grafici disponibili.
Si potrà scegliere tra vari tipi di:
\begin{itemize}
  \item \textit{Scatter Plot};
  \item \textit{Parallel Coordinates};
  \item \textit{Sankey Diagram}.
\end{itemize}

\begin{figure}[H]
    \includegraphics[width=1.0\textwidth]{ListaGrafici.jpg}
    \caption{Screenshot della lista dei grafici}
\end{figure}
Successivamente per visionare il grafico desiderato basterà premere il relativo bottone "Visualizza".

\subsection{Carica sessione}
Cliccando su "Carica sessione" comparirà una finestra di dialogo che ci permetterà di caricare i dati di una sessione salvata in precedenza. Seleziona quindi il file in formato .json desiderato e poi clicca su "Apri" per caricare il file.

\begin{figure}[H]
    \includegraphics[width=1.0\textwidth]{BottoneSessione.JPG}
    \caption{Screenshot della finestra di dialogo per il caricamento della sessione}
\end{figure}

Dopo aver caricato la sessione nel sistema si potrà proseguire il lavoro da dove era stato salvato.

\begin{figure}[H]
    \includegraphics[width=1.0\textwidth]{CaricamentoSessione.JPG}
    \caption{Screenshot della sessione caricata}
\end{figure}

\section{Bottoni}
Nella pagina di visualizzazione del grafico, sono disponibili tre bottoni. Questi si trovano nell'estremo superiore sinistro della pagina.

\begin{figure}[H]
    \includegraphics[width=1.0\textwidth]{Bottoni1.JPG}
    \caption{Screenshot che indica la posizione dei bottoni}
\end{figure}

\subsection{Bottone Home}
Questo bottone permette di ritornare alla Home in qualsiasi momento mantenendo comunque in memoria il dataset caricato, permettendo di scegliere un altra tipologia di grafico o di caricare un nuovo file.

\subsection{Nuovo Campionamento}
Ogni grafico attua un appropriato algoritmo di campionamento dei dati per permettere all'utilizzatore di avere una buona visualizzazione delle informazioni (il numero di dati potrebbe essere estremamente grande, quindi non gestibile).
Questo bottone permette di estrapolare ogni volta dei dati nuovi attraverso l'algoritmo di campionamento. Se il numero di dati del dataset è inferiore al numero massimo di dati che il grafico può visualizzare questo bottone non provoca cambiamenti.

\subsection{Salva Sessione}
Questo bottone permette di salvare la sessione corrente in tutti i suoi aspetti, compresi i filtri e dataset intero. Basterà premerlo per scaricare la sessione nel formato .json.

\section{Filtri}
In ogni grafico è possibile impostare dei filtri. Questi sono impostabili dall'estremo inferiore sinistro della pagina. Una volta inseriti i filtri desiderati occorre premere il bottone "Filtra" per applicarli. Per resettare i filtri invece occorre premere il bottone "Reset".

\begin{figure}[H]
    \includegraphics[width=1.0\textwidth]{Filtri.JPG}
    \caption{Screenshot che indica la posizione dei filtri}
\end{figure}

\subsection{Utente}
In questo campo è possibile inserire il numero di un utente se si vuole visualizzare il grafico contenente solo i suoi accessi.

\subsection{Ip}
In questo campo è possibile inserire un indirizzo IP se si vuole visualizzare il grafico contenente solo i suoi accessi.

\subsection{Evento}
Questo filtro permette di filtrare i dati secondo i vari tipi di evento, ovvero:
\begin{itemize}
  \item Login;
  \item Errore;
  \item Logout.
\end{itemize}

\subsection{Applicazione}
In questo campo è possibile inserire il nome di un'applicazione se si vuole visualizzare il grafico contenente solo gli accesi effettuati tramite essa.

\subsection{Data}
Qui viene fornito un calendario dal quale è possibile scegliere una data che permette di visualizzare gli accessi avvenuti in quella determinata giornata.

\section{Scatter Plot}
Il grafico \textit{Scatter Plot} permette di visualizzare ogni azione degli utenti sotto forma di punti all'interno del piano cartesiano.
Ogni evento ha un colore differente per avere una più facile visualizzazione.
Il verde indica un login effettuato con successo, il rosso un errore e il grigio un logout.
Per visualizzare le informazioni di ogni evento si dovrà posizionare il cursore al di sopra del pallino scelto. Le informazioni disponibili sono:
\begin{itemize}
  \item Ip;
  \item Numero Utente;
  \item Tipologia di evento;
  \item Data;
  \item Applicazione da dove è stata compiuta l'azione.
\end{itemize}

\begin{figure}[H]
    \includegraphics[width=1.0\textwidth]{VisualizzaInformazioni.JPG}
    \caption{Screenshot che mostra le informazioni di un punto}
\end{figure}

\section{Parallel Coordinates}
Il grafico \textit{Parallel Coordinates} permette di visualizzare ogni accesso sotto forma di linea.
Per visualizzare le informazioni di un particolare evento si dovrà posizionare il cursore al di sopra della linea scelta. Le informazioni disponibili sono:
\begin{itemize}
  \item Ip;
  \item Numero Utente;
  \item Tipologia di evento;
  \item Data;
  \item Applicazione da dove è stata compiuta l'azione.
\end{itemize}

\begin{figure}[H]
    \includegraphics[width=1.0\textwidth]{Coordinates.JPG}
    \caption{Screenshot che mostra le informazioni di una linea}
\end{figure}

\section{Sankey Diagram}
Il grafico \textit{Sankey Diagram} permette di visualizzare gli eventi raggruppati in nodi.
È per esempio possibile visualizzare quanti login ci sono stati nel mese di Aprile nell'orario d'ufficio.
Per visualizzare il numero di dati che ogni nodo contiene basterà posizionare il cursore al di sopra del rettangolo colorato che indica i dati che ci interessano.

\begin{figure}[H]
    \includegraphics[width=1.0\textwidth]{Sankey.JPG}
    \caption{Screenshot che mostra le informazioni di un nodo}
\end{figure}

\section{Sankey Diagram}
Il grafico \textit{Force-Directed Graph} permette di visualizzare i vari utenti con il loro rapporto Login/Errori.
Come indicato dalla legenda i nodi colorati di nero indicano i vari range di rapporto Login/Errori mentre i nodi con i vari colori indicano in che range l'utente si trova.
Posizionando il cursore sui nodi neri viene visualizzato il range a cui ci si riferisce, mentre posizionandolo su un nodo colorato vengono mostrate le informazioni rilevanti di quel determinato utente:
\begin{itemize}
  \item id;
  \item numero di login corretti;
  \item numero di login errati;
  \item percentuale relativa al rapporto Login/Errori.
\end{itemize}

\begin{figure}[H]
    \includegraphics[width=1.0\textwidth]{Force-Directed.JPG}
    \caption{Screenshot che mostra le informazioni di un utente}
\end{figure}
\begin{figure}[H]
    \centering
    \includegraphics[width=0.3\textwidth]{LegendaForce.JPG}
    \caption{Screenshot che mostra la legenda}
\end{figure}
