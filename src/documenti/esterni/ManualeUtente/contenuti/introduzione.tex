\chapter{Introduzione}
\section{Scopo del documento}
Questo documento ha lo scopo di illustrare le istruzioni per l’utilizzo e le funzionalità fornite dall’applicazione. L’utente sarà quindi a conoscenza dei requisiti minimi necessari per il corretto funzionamento di \textit{Login Warrior}, di come installarla in locale e di come farne un utilizzo consapevole.
\section{Scopo del Prodotto}
Al giorno d'oggi ogni servizio presente sul web richiede un'autenticazione$_G$ tramite login$_G$, fase fondamentale per la protezione dei dati di un individuo. Risulta ancora più importante se viene considerata la possibile presenza di malintenzionati con lo scopo di rubare ciò che dovrebbe essere privato. La presenza di attacchi informatici negli anni è andata aumentando e continua tuttora a crescere, per questo è necessario che questa pratica venga il più possibile riconosciuta e arginata.

Il capitolato C5 ha proprio come obiettivo quello di trovare una soluzione a questo problema. L'idea è quella di riconoscere le attività lecite e quelle illecite attraverso la raccolta, l'analisi e la visualizzazione di dati sotto forma di grafici e modelli che permettano un riconoscimento immediato delle differenze nei tentativi di accesso.

Con questo scopo il gruppo \textit{MERL} si impegnerà nella realizzazione di un'applicazione web$_G$ in grado di leggere grandi quantità di dati di login per poi mostrare tramite dei grafici la natura di questi, riuscendo nell'intento di riconoscere a primo impatto le attività sospette.
\section{Glossario}
Al fine di evitare incomprensioni relative alla terminologia usata all'interno del documento, viene fornito un Glossario nel file \textit{Glossario V2.0.0} in grado di dare una definizione precisa per ogni vocabolo potenzialmente ambiguo. Tali termini verranno evidenziati all'interno del documento con una G in pedice.
