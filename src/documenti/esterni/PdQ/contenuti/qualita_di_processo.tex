\chapter{Qualità di processo}
Per garantire un prodotto stabile e di qualità entro i costi e tempi stabiliti nel \textit{Piano di Progetto}, il gruppo \textit{MERL} ha deciso di adottare lo standard \textit{SPICE}$_G$. Questo standard garantisce la qualità di tutti i processi attraverso una definizione chiara degli obiettivi e di soglie minime prestabilite da rispettare.
Per quanto riguarda il miglioramento continuo nella qualità dei processi si è deciso di utilizzare il \textit{Ciclo di Deming}, questo garantisce una qualità tesa al miglioramento continuo dei processi e all'utilizzo ottimale delle risorse, e prevede una costante integrazione tra ricerca, progettazione, verifica$_G$ e produzione.

\section{Obiettivi}
\begin{table}[H]
  \renewcommand{\arraystretch}{1.25}
  \begin{tabular}{|p{2.5cm}|p{8cm}|p{1.7cm}|} \hline
    \rowcolor[HTML]{036400}
    \textcolor{white}{\textbf{Obiettivo}} & \textcolor{white}{\textbf{Descrizione}} & \textcolor{white}{\textbf{Metrica}}  \\ \hline
    \rowcolor[HTML]{EFEFEF}
    Budget & Evitare differenze eccessive rispetto al costo preventivato & MPC1 \newline MPC2 \newline MPC3     \\ \hline
    \rowcolor[HTML]{C0C0C0}
    Formazione & Ciascun componente del gruppo deve possedere un livello adeguato di preparazione, per cercare di evitare ritardi nella produzione   & \    \\ \hline
    \rowcolor[HTML]{EFEFEF}
    Calendario & Assicurare una pianificazione adatta ai compiti da svolgere, con conseguente massimizzazione dell'efficienza$_G$ della produzione  & MPC4       \\ \hline
  \end{tabular}
  \caption{Tabella degli obiettivi di qualità}
\end{table}

\newpage

\section{Metriche}

\begin{table}[H]
  \renewcommand{\arraystretch}{1.25}
  \begin{tabular}{|p{1.7cm}|p{3.5cm}|p{4cm}|p{3.3cm}|} \hline
    \rowcolor[HTML]{036400}
    \textcolor{white}{\textbf{Metrica}} & \textcolor{white}{\textbf{Nome}} & \textcolor{white}{\textbf{Valore Accettabile}} & \textcolor{white}{\textbf{Valore Ottimale}}    \\ \hline
      \rowcolor[HTML]{EFEFEF}
      MPC1 & Budget at Completion & Errore del +/- 5\% rispetto al preventivo & Corrispondente al preventivo \\ \hline
      \rowcolor[HTML]{C0C0C0}
      MPC2 & Budget Variance    &  +/- 10\% rispetto al preventivo & 0\% rispetto al preventivo     \\ \hline
      \rowcolor[HTML]{EFEFEF}
      MPC3 & Actual Cost    & Minore del budget totale  & Corrispondente al preventivo     \\ \hline
      \rowcolor[HTML]{C0C0C0}
      MPC4 & Schedule Variance & 7 giorni di ritardo/anticipo & 0 giorni di ritardo/anticipo     \\ \hline
  \end{tabular}
  \caption{Tabella delle metriche di qualità}
\end{table}
