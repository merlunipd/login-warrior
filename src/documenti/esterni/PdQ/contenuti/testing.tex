\chapter{Testing}
Per assicurarsi di raggiungere dei buoni livelli di qualità del prodotto, il gruppo \textit{MERL} ha deciso di eseguire la fase di test$_G$ in parallelo allo sviluppo delle varie componenti. In questo modo è possibile verificare che le parti di programma sottoposte a controllo siano implementate correttamente e assumano un comportamento atteso.
\section{Tipologie di test}
    \subsection{Test di Unità (TU)}  Servono a verificare che le più piccole pati di programma prese singolarmente abbiano un funzionamento autonomo.
    \subsection{Test di Integrazione (TI)} Ha la funzione di verificare che le singole unità interagiscano tra loro nel modo corretto.
    \subsection{Test di Regressione (TR)} Verificano che l'implementazione di nuove componenti dell'applicativo non generino nuovi errori.
    \subsection{Test di Sistema (TS)} Verificano che il comportamento dell'intero sistema sia conforme a quanto prestabilito in termini prestazionali e verificano la corretta implementazione dei requisiti.
    \subsection{Test di Accettazione (TA)} Svolti con il committente$_G$, hanno la funzione di verificare che il prodotto
        finale sia completo, funzionante e rispetti le caratteristiche concordate tra le parti.


\section{Specifica dei test}
\begin{center}
    \renewcommand\arraystretch{1.5}
    \centering
    \begin{longtable}{|p{1.5cm}|p{11cm}|p{1cm}|}
    \hline
    \rowcolor[HTML]{036400}
    \textcolor{white}{\textbf{Codice}} & \textcolor{white}{\textbf{Descrizione}} & \textcolor{white}{\textbf{Stato}} \\ \hline
        \rowcolor[HTML]{EFEFEF}
        STF1 & L'utente deve poter caricare dati nel sistema tramite file CSV$_G$ oppure caricare una sessione precedente tramite un file JSON$_G$. \`E necessario verificare che: \begin{itemize}
            \item L'utente possa caricare i dati mediante gli appositi bottoni;
            \item I file caricati siano sintatticamente corretti;
            \item Sia visualizzato a schermo un messaggio con l'esito dell'operazione.
        \end{itemize} & NI\\ \hline
        \rowcolor[HTML]{C0C0C0}
        STF2 & L'utente deve essere informato in caso di errore nel caricamento dei file. \'E necessario verificare che: \begin{itemize}
            \item L'avviso di errore sia chiaro e ben visibile;
            \item Sia possibile inserire nuovamente i file.
        \end{itemize} & NI\\ \hline
        \rowcolor[HTML]{EFEFEF}
        STF3.1 & L'applicazione deve fornire la scelta del grafico. Verificare che: \begin{itemize}
            \item Sia disponibile la scelta del grafico \textit{Scatter Plot}$_G$;
            \item Sia disponibile la scelta del grafico \textit{Parallel Coordinates}$_G$;
            \item Sia disponibile la scelta del grafico \textit{Force-Direct Graph}$_G$;
            \item Sia disponibile la scelta del grafico \textit{Sankey Diagram}$_G$;
            \item La scelta dell'utente venga elaborata correttamente dall'applicazione.
        \end{itemize} & NI\\ \hline
        \rowcolor[HTML]{C0C0C0}
        STF3.2 & L'applicazione deve fornire la configurazione del grafico \textit{Scatter Plot}. Verificare che: \begin{itemize}
            \item L'utente possa scegliere quali combinazioni di dimensioni l'applicativo deve visualizzare.
        \end{itemize} & NI\\ \hline
        \rowcolor[HTML]{EFEFEF}
        STF3.3 & L'applicazione deve fornire la configurazione del grafico \textit{Parallel Coordinates}. Verificare che: \begin{itemize}
            \item L'utente possa scegliere quali combinazioni di dimensioni l'applicativo deve visualizzare.
        \end{itemize} & NI\\ \hline
        \rowcolor[HTML]{C0C0C0}
        STF3.4 & L'applicazione deve fornire la configurazione del grafico \textit{Force-Direct Graph}. Verificare che: \begin{itemize}
            \item L'utente possa scegliere quali combinazioni di dimensioni l'applicativo deve visualizzare.
        \end{itemize} & NI\\ \hline
        \rowcolor[HTML]{EFEFEF}
        STF3.5 & L'applicazione deve fornire la configurazione del grafico \textit{Sankey Diagram}. Verificare che: \begin{itemize}
            \item L'utente possa scegliere quali combinazioni di dimensioni l'applicativo deve visualizzare.
        \end{itemize} & NI\\ \hline
        \rowcolor[HTML]{C0C0C0}
        STF4.1 & L'applicazione deve permettere di modificare il grafico \textit{Scatter Plot}. Verificare che: \begin{itemize}
            \item Sia possibile modificare la visualizzazione dei punti;
            \item Sia possibile modificare i colori;
            \item La rappresentazione grafica sia la più chiara possibile;
            \item Le modifiche vengano visualizzate correttamente.
        \end{itemize} & NI\\ \hline
        \rowcolor[HTML]{EFEFEF}
        STF4.2 & L'applicazione deve permettere di modificare il grafico \textit{Parallel Coordinates}. Verificare che: \begin{itemize}
            \item Sia possibile modificare l'opacità delle curve del grafico;
            \item Sia possibile modificare la curvatura delle linee del grafico;
            \item Sia possibile modificare la forza di raggruppamento delle linee del grafico;
            \item La rappresentazione grafica sia la più chiara possibile;
            \item Le modifiche vengano visualizzate correttamente.
        \end{itemize} & NI\\ \hline
        \rowcolor[HTML]{C0C0C0}
        STF4.3 & L'applicazione deve permettere di modificare il grafico \textit{Force-Direct Graph}. Verificare che: \begin{itemize}
            \item Sia possibile modificare l'intensità della forza di repulsione del grafico;
            \item Sia possibile modificare l'intensità della forza di tensione del grafico;
            \item Sia possibile modificare i colori al grafico;
            \item La rappresentazione grafica sia la più chiara possibile;
            \item Le modifiche vengano visualizzate correttamente.
        \end{itemize} & NI\\ \hline
        \rowcolor[HTML]{EFEFEF}
        STF4.4 & L'applicazione deve permettere di modificare il grafico \textit{Sankey Diagram}. Verificare che: \begin{itemize}
            \item Sia possibile modificare i colori dei link del grafico;
            \item Sia possibile modificare l'opacità dei link del grafico;
            \item Sia possibile modificare l'allineamento dei nodi del grafico ;
            \item La rappresentazione grafica sia la più chiara possibile;    
            \item Le modifiche vengano visualizzate correttamente.
        \end{itemize} & NI\\ \hline
        \rowcolor[HTML]{C0C0C0}
        STF5 & L'applicazione deve permettere di impostare vari filtri. Verificare che: \begin{itemize}
            \item Sia possibile impostare un filtro sui dati;
            \item Sia possibile impostare un filtro sugli utenti;
            \item Sia possibile impostare un filtro sul tipo di evento;
            \item Sia possibile impostare un filtro sulla data;
            \item Sia possibile impostare un filtro sul tipo di applicazione;
            \item Il grafico rispetti le personalizzazioni scelte;
            \item Sia possibile modificare i parametri scelti;
            \item Le modifiche vengano visualizzate correttamente.
        \end{itemize} & NI\\ \hline
        \rowcolor[HTML]{EFEFEF}
        STF6 & L'applicazione deve permettere la visualizzazione in qualunque momento del manuale utente. Verificare che: \begin{itemize}
            \item L'utente possa individuare e consultare facilmente la guida;
            \item La guida sia comprensibile a qualunque tipologia di utente;
            \item La guida descriva tutti gli utilizzi e le caratteristiche dell'applicazione.
        \end{itemize} & NI\\ \hline
        \rowcolor[HTML]{C0C0C0}
        STF7 & L'applicazione deve permettere il salvataggio della configurazione$_G$ e delle impostazioni selezionate per permettere di ripristinarle in un momento successivo. Verificare che: \begin{itemize}
            \item Sia disponibile un bottone per il salvataggio della sessione;
            \item La sessione venga salvata correttamente;
        \end{itemize}& NI\\ \hline
        \caption{Tabella della specifica dei test}
    \end{longtable}
\end{center}

    Attualmente lo stato di tutti i test è contrassegnato dalla sigla \textit{NI} che indica "Non Implementato".
\begin{center}
    \renewcommand\arraystretch{1.5}
    \centering
    \begin{longtable}{|p{3cm}|p{10cm}|} \hline
        \rowcolor[HTML]{036400}
        \textcolor{white}{\textbf{Codice Test}} & \textcolor{white}{\textbf{Codice Requisiti}} \\ \hline
        \rowcolor[HTML]{EFEFEF}
        STF1 & RF.1.1 RF.2.3\\ \hline
        \rowcolor[HTML]{C0C0C0}
        STF2 & RF.1.2 RF.1.4 \\ \hline
        \rowcolor[HTML]{EFEFEF}
        STF3.1 & RF.1.5.1 RF.1.5.2 RF.1.5.3 RF.1.5.4 \\ \hline
        \rowcolor[HTML]{C0C0C0}
        STF3.2 & RF.1.5.5 \\ \hline
        \rowcolor[HTML]{EFEFEF}
        STF3.3 & RF.1.5.6 \\ \hline
        \rowcolor[HTML]{C0C0C0}
        STF3.4 & RF.1.5.7 \\ \hline
        \rowcolor[HTML]{EFEFEF}
        STF3.5 & RF.1.5.8 \\ \hline
        \rowcolor[HTML]{C0C0C0}
        STF4.1 & RF.2.6.1.1 RF.2.6.1.2\\ \hline
        \rowcolor[HTML]{EFEFEF}
        STF4.2 & RF.2.6.2.1 RF.2.6.2.2 RF.2.6.2.3\\ \hline
        \rowcolor[HTML]{C0C0C0}
        STF4.3 & RF.2.6.3.1 RF.2.6.3.2 RF.2.6.3.3\\ \hline
        \rowcolor[HTML]{EFEFEF}
        STF4.4 & RF.2.6.4.1 RF.2.6.4.2 RF.2.6.4.3\\ \hline
        \rowcolor[HTML]{C0C0C0}
        STF5 & RF.2.7 RF.2.7.1 RF.2.7.2 RF.2.7.3 RF.2.7.4 RF.2.7.5 \\ \hline
        \rowcolor[HTML]{EFEFEF}
        STF6 & RF.2.8 \\ \hline
        \rowcolor[HTML]{C0C0C0}
        STF7 & RF.2.9 \\ \hline
        \caption{Tracciamento test - requisiti funzionali}
    \end{longtable}
\end{center}
