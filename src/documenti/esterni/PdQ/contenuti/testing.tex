\chapter{Testing}
Per assicurarsi di raggiungere dei buoni livelli di qualità per il prodotto, il gruppo MERL 
ha adottato come modello di sviluppo il \textbf{Modello a V}. In questo modo è possibile verificare 
che le parti di programma sottoposte a controllo siano implementate correttamente e assumano un 
comportamento atteso.
\\Possiamo individuare differenti tipologie di test:
    \begin{itemize}
        \item \textbf{Test di Unità}
        \item \textbf{Test di Integrazione} 
        \item \textbf{Test di Sistema}
        \item \textbf{Test di Accettazione}
        
    \end{itemize}

\section{Tipologie di test}
    \subsection{Test di Unità (TU)}  Servono a verificare che le più piccole pati di programma prese singolarmente abbiano un funzionamento autonomo.
    \subsection{Test di Integrazione (TI)} Ha la funzione di verificare che le singole unità interagiscano tra loro nel modo corretto.
    \subsection{Test di Sistema (TS)} Verificano che il comportamento dell'intero sistema sia conforme a quanto prestabilito in termini prestazionali e verificano la coretta implementazione dei requisiti.
    \subsection{Test di Accettazione (TA)} Svolti con il committente, hanno la funzione di verificare che il prodotto 
        finale sia completo, funzionante e rispetti le caratteristiche concordate tra le parti.


\section{Specifica dei test}
    \begin{tabular}{|p{2cm}|p{4cm}|p{4cm}|} \hline
      \textbf{codice} & \textbf{Descrizione} & \textbf{Stato} \\ \hline
        TS1F1 & L'utente deve poter caricare dati nel sistema tramite file CSV oppure caricare una sessione precedente tramite un file .json. \'E necessario verificare che: \begin{itemize}
            \item L'utente possa caricare i dati mediante gli appositi bottoni;
            \item I file caricati siano sintatticamente corretti;
            \item Sia visualizzato a schermo un messaggio con l'esito dell'operazione;
            \item In caso di errore sia possibile inserire nuovamente i file.
        \end{itemize} & NI\\ \hline
        
        TS1F2 & L'utente deve poter visualizzare con quali dimensioni è possibile visualizzare il grafico. Verificare che: \begin{itemize}
            \item Siano visualizzate le dimensioni corrette;
            \item Ogni dimensione sia selezionabile;
            \item Le dimensioni abbiano un'effettiva utilità per l'utente.
        \end{itemize} & NI\\ \hline

        TS1F3 & L'utente deve poter decidere con quale visualizzazione l'applicativo rapresenta i dati. Verificare che: \begin{itemize}
            \item Sia visualizzata in modo intuitivo la scelta del grafico da utilizzare.
        \end{itemize} & NI\\ \hline

        TS1F3.1 & L'applicazione deve fornire la visualizzazione  \textit{Scatter Plot}. Verificare che: \begin{itemize}
            \item L'utente possa scegliere quali combinazioni di dimensioni l'applicatico deve visualizzare;
            \item Ogni scatter plot sia corretto rispetto ai dati scelti da rappresentare;
        \end{itemize} & NI\\ \hline

        TS1F3.2 & L'applicazione deve fornire la visualizzazione  \textit{Parallel Coordinates}. Verificare che: \begin{itemize}
            \item La rappresentazione grafica sia la più chiara possibile;
            \item Ogni caratteristica interessante sia rappresentata negli assi.
        \end{itemize} & NI\\ \hline

        TS1F3.3 & L'applicazione deve fornire la visualizzazione  \textit{Force-Direct Graph}. Verificare che: \begin{itemize}
            \item Ogni linea rappresenti una relazione tra i nodi;
            \item Le distanze e le forze di attrazione e repulsione tra i nodi siano correttamente implementate;
            \item L'utente possa interagire con il grafico.
        \end{itemize} & NI\\ \hline

        TS1F3.4 & L'applicazione deve fornire la visualizzazione  \textit{Sankey Diagram}. Verificare che: \begin{itemize}
            \item La rappresentazione grafica sia la più chiara possibile;
            \item Sia possibile interagire con il grafico andando a risposizionare le barre verticali.
        \end{itemize} & NI\\ \hline

        TS1F4 & L'applicazione deve mettere a disposiazione alcuni parametri per personalizzare la visualizzazione. Vericare che: \begin{itemize}
            \item L'utente possa scegliere le caratteristiche messe a disposizione;
            \item Le caratteristiche vengono applicate e visualizzate correttamente.
        \end{itemize} & NI\\ \hline

        TS1F5 & L'applicazione deve permettere la visualizzazione in qualunque momento del manuale utente. Verificare che: \begin{itemize}
            \item L'utente possa individuare e consultare facilmente la guida;
            \item La guida sia comprensibile a qualunque tipologia di utente;
            \item La guida descriva tutti gli utilizzi e le caratteristiche dell'applicazione.
        \end{itemize}& NI\\ \hline
        & & \\ \hline

        TS1F6 & L'applicazione deve permettere il salvataggio della configurazione, delle impostazioni selezionate e permettere di ripristinarle in un momento successivo. Verificare che: \begin{itemize}
            \item Sia disponibile un bottone per il salvataggio della sessione;
            \item Sia disponibile un bottone per il caricamento di una sessione precedente;
            \item La sessione venga salvata correttamente;
            \item La sessione vega caricata correttamente;
        \end{itemize}& NI\\ \hline
    \end{tabular}
