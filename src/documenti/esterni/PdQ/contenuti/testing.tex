\chapter{Testing}
Per assicurarsi di raggiungere dei buoni livelli di qualità del prodotto, il gruppo \textit{MERL} ha deciso di eseguire la fase di test$_G$ in parallelo allo sviluppo delle varie componenti. In questo modo è possibile verificare che le parti di programma sottoposte a controllo siano implementate correttamente e assumano un comportamento atteso.
\section{Tipologie di test}
    \subsection{Test di Unità (TU)}  Servono a verificare che le più piccole pati di programma prese singolarmente abbiano un funzionamento autonomo.
    \begin{center}
        \renewcommand\arraystretch{1.5}
        \centering
        \begin{longtable}{|p{1.5cm}|p{11cm}|p{1cm}|}
        \hline
        \rowcolor[HTML]{036400}
        \textcolor{white}{\textbf{Codice}} & \textcolor{white}{\textbf{Descrizione}} & \textcolor{white}{\textbf{Stato}} \\ \hline
            \rowcolor[HTML]{EFEFEF}
            TU1 & Verificare che i dati inseriti siano caricati nel sistema. & I\\ \hline
            \rowcolor[HTML]{C0C0C0}
            TU2 & Verificare che venga visualizzato un messaggio d'errore se i dati non sono stati inseriti correttamente. & I\\ \hline
            \rowcolor[HTML]{EFEFEF}
            TU3 & Verificare che la lettura del file CSV avvenga correttamente. & I\\ \hline
            \rowcolor[HTML]{C0C0C0}
            TU4 & Verificare che la lettura del file JSON avvenga correttamente. & I\\ \hline
            \rowcolor[HTML]{C0C0C0}
            TU5 & Verificare che il grafico Scatter Plot venga renderizzato correttamente. & I\\ \hline
            \rowcolor[HTML]{EFEFEF}
            TU6 & Verificare che il grafico Parallel Coordinates venga renderizzato correttamente. & I\\ \hline
            \rowcolor[HTML]{C0C0C0}
            TU7 & Verificare che il grafico Force-Direct Graph venga renderizzato correttamente. & I\\ \hline
            \rowcolor[HTML]{EFEFEF}
            TU8 & Verificare che il grafico Sankey Diagram venga renderizzato correttamente. & I\\ \hline
            \rowcolor[HTML]{C0C0C0}
            TU9 & Verificare che le configurazioni del grafico Scatter Plot siano visualizzate correttamente nel sistema. & I\\ \hline
            \rowcolor[HTML]{EFEFEF}
            TU10 & Verificare che le configurazioni del grafico Parallel Coordinates siano visualizzate correttamente nel sistema & I\\ \hline
            \rowcolor[HTML]{C0C0C0}
            TU11 & Verificare che le configurazioni del grafico Force-Direct Graph siano visualizzate correttamente nel sistema. & I\\ \hline
            \rowcolor[HTML]{EFEFEF}
            TU12 & Verificare che le configurazioni del grafico Sankey Diagram siano visualizzate correttamente nel sistema. & I\\ \hline
            \rowcolor[HTML]{C0C0C0}
            TU13 & Verificare che le modifiche del grafico Scatter Plot vengano visualizzate correttamente. & NI\\ \hline
            \rowcolor[HTML]{EFEFEF}
            TU14 & Verificare che le modifiche del grafico Parallel Coordinates vengano visualizzate correttamente. & NI\\ \hline
            \rowcolor[HTML]{C0C0C0}
            TU15 & Verificare che le modifiche del grafico Force-Direct Graph vengano visualizzate correttamente. & NI\\ \hline
            \rowcolor[HTML]{EFEFEF}
            TU16 & Verificare che le modifiche del grafico Sankey Diagram vengano visualizzate correttamente. & NI\\ \hline
            \caption{Tabella dei test di unità}
        \end{longtable}
    \end{center}

    \subsection{Test di Integrazione (TI)} Ha la funzione di verificare che le singole unità interagiscano tra loro nel modo corretto.
    \begin{center}
        \renewcommand\arraystretch{1.5}
        \centering
        \begin{longtable}{|p{1.5cm}|p{11cm}|p{1cm}|}
        \hline
        \rowcolor[HTML]{036400}
        \textcolor{white}{\textbf{Codice}} & \textcolor{white}{\textbf{Descrizione}} & \textcolor{white}{\textbf{Stato}} \\ \hline
            \rowcolor[HTML]{EFEFEF}
            TI1 & Verificare che il collegamento con il database avvenga correttamente. & I\\ \hline
            \rowcolor[HTML]{C0C0C0}
            TI2 & Verificare il dataset presente nel database sia raggiungibile. & I\\ \hline
            \rowcolor[HTML]{EFEFEF}
            TI3 & Verificare che la chiusura del collegamento con il database avvenga correttamente. & I\\ \hline
            \rowcolor[HTML]{C0C0C0}
            TI4 & Verificare che l’integrazione con la libreria di visualizzazione dei grafici sia gestita correttamente. & I\\ \hline   
            \caption{Tabella dei test di integrazione}
        \end{longtable}
    \end{center}

    \subsection{Test di Regressione (TR)} Verificano che l'implementazione di nuove componenti dell'applicativo non generino nuovi errori.
    \begin{center}
        \renewcommand\arraystretch{1.5}
        \centering
        \begin{longtable}{|p{1.5cm}|p{11cm}|p{1cm}|}
        \hline
        \rowcolor[HTML]{036400}
        \textcolor{white}{\textbf{Codice}} & \textcolor{white}{\textbf{Descrizione}} & \textcolor{white}{\textbf{Stato}} \\ \hline
            \rowcolor[HTML]{C0C0C0}
            TR1 & Verificare che i dati vengano inseriti correttamente nel sistema. & I\\ \hline
            \rowcolor[HTML]{EFEFEF}
            TR2 & Verificare che venga visualizzato a schermo un messaggio d’esito dell’operazione di caricamento di un file CSV.& I\\ \hline
            \rowcolor[HTML]{C0C0C0}
            TR3 & Verificare che venga visualizzato a schermo un messaggio d’esito dell’operazione di caricamento di un file JSON.& I\\ \hline
            \rowcolor[HTML]{EFEFEF}
            TR4 & Verificare che l’utente possa selezionare la configurazione del grafico Scatter Plot. & I\\ \hline
            \rowcolor[HTML]{C0C0C0}
            TR5 & Verificare che l’utente possa selezionare la configurazione del grafico Parallel Coordinates. & I\\ \hline
            \rowcolor[HTML]{EFEFEF}
            TR6 & Verificare che l’utente possa selezionare la configurazione del grafico Force-Direct Graph. & I\\ \hline
            \rowcolor[HTML]{C0C0C0}
            TR7 & Verificare che l’utente possa selezionare la configurazione del grafico Sankey Diagram. & I\\ \hline
            \rowcolor[HTML]{EFEFEF}
            TR8 & Verificare che l’utente possa selezionare la configurazione del grafico Scatter Plot. & I\\ \hline
            \rowcolor[HTML]{C0C0C0}
            TR9 & Verificare che l’utente possa apportare modifiche al grafico Scatter Plot. & NI\\ \hline
            \rowcolor[HTML]{EFEFEF}
            TR10 & Verificare che l’utente possa apportare modifiche al grafico Parallel Coordinates. & NI\\ \hline
            \rowcolor[HTML]{C0C0C0}
            TR11 & Verificare che l’utente possa apportare modifiche al grafico Force-Direct Graph. & NI\\ \hline
            \rowcolor[HTML]{EFEFEF}
            TR12 & Verificare che l’utente possa apportare modifiche al grafico Sankey Diagram. & NI\\ \hline
            \rowcolor[HTML]{C0C0C0}
            TR13 & Verificare che l’utente possa applicare i filtri al grafico Scatter Plot. & I\\ \hline
            \rowcolor[HTML]{EFEFEF}
            TR14 & Verificare che l’utente possa applicare i filtri al grafico Parallel Coordinates. & I\\ \hline
            \rowcolor[HTML]{C0C0C0}
            TR15 & Verificare che l’utente possa applicare i filtri al grafico Force-Direct Graph. & NI\\ \hline
            \rowcolor[HTML]{EFEFEF}
            TR16 & Verificare che l’utente possa applicare i filtri al grafico Sankey Diagram. & I\\ \hline
            \rowcolor[HTML]{C0C0C0}
            TR17 & Verificare che l’utente possa consultare il manuale. & NI\\ \hline
            \rowcolor[HTML]{EFEFEF}
            TR18 & Verificare che l’utente possa salvare la sessione di lavoro in corso o ripristinarne una precedente. & I\\ \hline
            \rowcolor[HTML]{C0C0C0}
            TR19 & Verificare che l'applicazione sia compatibile con il browser \textit{Chrome} dalla versione 61. & I\\ \hline
            \rowcolor[HTML]{EFEFEF}
            TR20 & Verificare che l'applicazione sia compatibile con il browser \textit{Edge} dalla versione 16. & I\\ \hline
            \rowcolor[HTML]{C0C0C0}
            TR21 & Verificare che l'applicazione sia compatibile con il browser \textit{Firefox} dalla versione 60. & I\\ \hline
            \rowcolor[HTML]{EFEFEF}
            TR22 & Verificare che l'applicazione sia compatibile con il browser \textit{Opera} dalla versione 48. & I\\ \hline
            \rowcolor[HTML]{C0C0C0}
            TR23 & Verificare che l'applicazione sia compatibile con il browser \textit{Safari} dalla versione 10.1. & I\\ \hline
            \caption{Tabella dei test di regressione}
        \end{longtable}
    \end{center}


    \subsection{Test di Sistema (TS)} Verificano che il comportamento dell'intero sistema sia conforme a quanto prestabilito in termini prestazionali e verificano la corretta implementazione dei requisiti.
    \begin{center}
        \renewcommand\arraystretch{1.5}
        \centering
        \begin{longtable}{|p{1.5cm}|p{11cm}|p{1cm}|}
        \hline
        \rowcolor[HTML]{036400}
        \textcolor{white}{\textbf{Codice}} & \textcolor{white}{\textbf{Descrizione}} & \textcolor{white}{\textbf{Stato}} \\ \hline
            \rowcolor[HTML]{C0C0C0}
            TS1F1 & Verificare che l’utente possa caricare dei dati nel sistema tramite file CSV. & I\\ \hline
            \rowcolor[HTML]{EFEFEF}
            TS1F2 & Verificare che l’utente possa caricare dei dati nel sistema tramite file JSON. & I\\ \hline
            \rowcolor[HTML]{C0C0C0}
            TS1F3 & Verificare che all’utente venga visualizzato a schermo un messaggio d’esito dell’operazione di caricamento di un file CSV.& I\\ \hline
            \rowcolor[HTML]{EFEFEF}
            TS1F4 & Verificare che all’utente venga visualizzato a schermo un messaggio d’esito dell’operazione di caricamento di un file JSON.& I\\ \hline
            \rowcolor[HTML]{C0C0C0}
            TS1F5 & Verificare che l’utente possa selezionare la configurazione del grafico Scatter Plot. & I\\ \hline
            \rowcolor[HTML]{EFEFEF}
            TS1F6 & Verificare che l’utente possa selezionare la configurazione del grafico Parallel Coordinates. & I\\ \hline
            \rowcolor[HTML]{C0C0C0}
            TS1F7 & Verificare che l’utente possa selezionare la configurazione del grafico Force-Direct Graph. & I\\ \hline
            \rowcolor[HTML]{EFEFEF}
            TS1F8 & Verificare che l’utente possa selezionare la configurazione del grafico Sankey Diagram. & I\\ \hline
            \rowcolor[HTML]{C0C0C0}
            TS2F9 & Verificare che l’utente possa apportare modifiche al grafico Scatter Plot. & NI\\ \hline
            \rowcolor[HTML]{EFEFEF}
            TS2F10 & Verificare che l’utente possa apportare modifiche al grafico Parallel Coordinates. & NI\\ \hline
            \rowcolor[HTML]{C0C0C0}
            TS2F11 & Verificare che l’utente possa apportare modifiche al grafico Force-Direct Graph. & NI\\ \hline
            \rowcolor[HTML]{EFEFEF}
            TS2F12 & Verificare che l’utente possa apportare modifiche al grafico Sankey Diagram. & NI\\ \hline
            \rowcolor[HTML]{C0C0C0}
            TS2F13 & Verificare che l’utente possa applicare i filtri al grafico Scatter Plot. & I\\ \hline
            \rowcolor[HTML]{EFEFEF}
            TS2F14 & Verificare che l’utente possa applicare i filtri al grafico Parallel Coordinates. & I\\ \hline
            \rowcolor[HTML]{C0C0C0}
            TS2F15 & Verificare che l’utente possa applicare i filtri al grafico Force-Direct Graph. & NI\\ \hline
            \rowcolor[HTML]{EFEFEF}
            TS2F16 & Verificare che l’utente possa applicare i filtri al grafico Sankey Diagram. & I\\ \hline
            \rowcolor[HTML]{C0C0C0}
            TS2F17 & Verificare che l’utente possa consultare il manuale. & NI\\ \hline
            \rowcolor[HTML]{EFEFEF}
            TS2F18 & Verificare che l’utente possa salvare la sessione di lavoro in corso o ripristinarne una precedente. & I\\ \hline
            \rowcolor[HTML]{C0C0C0}
            TS1F19 & Verificare che l'applicazione sia compatibile con il browser \textit{Chrome} dalla versione 61. & I\\ \hline
            \rowcolor[HTML]{EFEFEF}
            TS1F20 & Verificare che l'applicazione sia compatibile con il browser \textit{Edge} dalla versione 16. & I\\ \hline
            \rowcolor[HTML]{C0C0C0}
            TS1F21 & Verificare che l'applicazione sia compatibile con il browser \textit{Firefox} dalla versione 60. & I\\ \hline
            \rowcolor[HTML]{EFEFEF}
            TS1F22 & Verificare che l'applicazione sia compatibile con il browser \textit{Opera} dalla versione 48. & I\\ \hline
            \rowcolor[HTML]{C0C0C0}
            TS1F23 & Verificare che l'applicazione sia compatibile con il browser \textit{Safari} dalla versione 10.1. & I\\ \hline
            \caption{Tabella dei test di sistema}
        \end{longtable}
    \end{center}

    \subsection{Tracciamento dei test di sistema }
    \begin{center}
        \renewcommand\arraystretch{1.5}
        \centering
        \begin{longtable}{|p{1.5cm}|p{11cm}|p{1cm}|}
        \hline
        \rowcolor[HTML]{036400}
        \textcolor{white}{\textbf{Codice}} & \textcolor{white}{\textbf{Requisito}} \\ \hline
            \rowcolor[HTML]{EFEFEF}
            TS1F1, TS1F2 & RF.1.1\\ \hline
            \rowcolor[HTML]{C0C0C0}
            TS1F3, TS1F4  & RF.1.2\\ \hline
            \rowcolor[HTML]{EFEFEF}
            TS1F5 & RF.1.5.1, RF.1.5.5\\ \hline
            \rowcolor[HTML]{C0C0C0}
            TS1F6 & RF.1.5.2, RF.1.5.6\\ \hline
            \rowcolor[HTML]{EFEFEF}
            TS1F7 & RF.1.5.3, RF.1.5.7\\ \hline
            \rowcolor[HTML]{C0C0C0}
            TS1F8 & RF.1.5.4, RF.1.5.8\\ \hline
            \rowcolor[HTML]{EFEFEF}
            TS2F9 & RF.2.6.1.1, RF.2.6.1.2\\ \hline
            \rowcolor[HTML]{C0C0C0}
            TS2F10 & RF.2.6.2.1, RF.2.6.2.2, RF.2.6.2.3 \\ \hline
            \rowcolor[HTML]{EFEFEF}
            TS2F11 & RF.2.6.3.1, RF.2.6.3.2, RF.2.6.3.3 \\ \hline
            \rowcolor[HTML]{C0C0C0}
            TS2F12 & RF.2.6.4.1, RF.2.6.4.2, RF.2.6.4.3\\ \hline
            \rowcolor[HTML]{EFEFEF}
            TS1F13 & RF.2.7, RF.2.7.1, RF.2.7.2, RF.2.7.3, RF.2.7.4, RF.2.7.5\\ \hline
            \rowcolor[HTML]{C0C0C0}
            TS1F14 & RF.2.7, RF.2.7.1, RF.2.7.2, RF.2.7.3, RF.2.7.4, RF.2.7.5\\ \hline
            \rowcolor[HTML]{EFEFEF}
            TS1F15 & RF.2.7, RF.2.7.1, RF.2.7.2, RF.2.7.3, RF.2.7.4, RF.2.7.5\\ \hline
            \rowcolor[HTML]{C0C0C0}
            TS1F16 & RF.2.7, RF.2.7.1, RF.2.7.2, RF.2.7.3, RF.2.7.4, RF.2.7.5\\ \hline
            \rowcolor[HTML]{EFEFEF}
            TS1F17 & RF.2.8\\ \hline
            \rowcolor[HTML]{C0C0C0}
            TS2F18 & RF.2.3, RF.2.9\\ \hline
            \rowcolor[HTML]{EFEFEF}
            TS1F19 & RV.1.5\\ \hline
            \rowcolor[HTML]{C0C0C0}
            TS1F20 & RV.1.6\\ \hline
            \rowcolor[HTML]{EFEFEF}
            TS1F21 & RV.1.7\\ \hline
            \rowcolor[HTML]{C0C0C0}
            TS1F22 & RV.1.8\\ \hline
            \rowcolor[HTML]{EFEFEF}
            TS1F23 & RV.1.9\\ \hline
            \caption{Tabella del tracciamento dei test di sistema }
        \end{longtable}
    \end{center}


    \subsection{Test di Accettazione (TA)} Svolti con il committente$_G$, hanno la funzione di verificare che il prodotto
        finale sia completo, funzionante e rispetti le caratteristiche concordate tra le parti.
        \begin{center}
            \renewcommand\arraystretch{1.5}
            \centering
            \begin{longtable}{|p{1.5cm}|p{11cm}|p{1cm}|}
            \hline
            \rowcolor[HTML]{036400}
            \textcolor{white}{\textbf{Codice}} & \textcolor{white}{\textbf{Descrizione}} & \textcolor{white}{\textbf{Stato}} \\ \hline
                \rowcolor[HTML]{EFEFEF}
                TA1 & Verificare che l’utente possa caricare dei dati nel sistema tramite file CSV. & I\\ \hline
                \rowcolor[HTML]{C0C0C0}
                TA2 & Verificare che l’utente possa caricare dei dati nel sistema tramite file JSON. & I\\ \hline
                \rowcolor[HTML]{EFEFEF}
                TA3 & Verificare che l’utente possa visualizzare a schermo un messaggio d’esito dell’operazione di caricamento di un file CSV.& I\\ \hline
                \rowcolor[HTML]{C0C0C0}
                TA4 & Verificare che l’utente possa visualizzare a schermo un messaggio d’esito dell’operazione di caricamento di un file JSON.& I\\ \hline
                \rowcolor[HTML]{EFEFEF}
                TA5 & Verificare che l’utente possa selezionare la configurazione del grafico Scatter Plot. & I\\ \hline
                \rowcolor[HTML]{C0C0C0}
                TA6 & Verificare che l’utente possa selezionare la configurazione del grafico Parallel Coordinates. & I\\ \hline
                \rowcolor[HTML]{EFEFEF}
                TA7 & Verificare che l’utente possa selezionare la configurazione del grafico Force-Direct Graph. & I\\ \hline
                \rowcolor[HTML]{C0C0C0}
                TA8 & Verificare che l’utente possa selezionare la configurazione del grafico Sankey Diagram. & I\\ \hline
                \rowcolor[HTML]{EFEFEF}
                TA9 & Verificare che l’utente possa selezionare la configurazione del grafico Scatter Plot. & I\\ \hline
                \rowcolor[HTML]{C0C0C0}
                TA10 & Verificare che l’utente possa apportare modifiche al grafico Scatter Plot. & NI\\ \hline
                \rowcolor[HTML]{EFEFEF}
                TA11 & Verificare che l’utente possa apportare modifiche al grafico Parallel Coordinates. & NI\\ \hline
                \rowcolor[HTML]{C0C0C0}
                TA12 & Verificare che l’utente possa apportare modifiche al grafico Force-Direct Graph. & NI\\ \hline
                \rowcolor[HTML]{EFEFEF}
                TA13 & Verificare che l’utente possa apportare modifiche al grafico Sankey Diagram. & NI\\ \hline
                \rowcolor[HTML]{C0C0C0}
                TA14 & Verificare che l’utente possa applicare i filtri al grafico Scatter Plot. & I\\ \hline
                \rowcolor[HTML]{EFEFEF}
                TA15 & Verificare che l’utente possa applicare i filtri al grafico Parallel Coordinates. & I\\ \hline
                \rowcolor[HTML]{C0C0C0}
                TA16 & Verificare che l’utente possa applicare i filtri al grafico Force-Direct Graph. & NI\\ \hline
                \rowcolor[HTML]{EFEFEF}
                TA17 & Verificare che l’utente possa applicare i filtri al grafico Sankey Diagram. & I\\ \hline
                \rowcolor[HTML]{C0C0C0}
                TA18 & Verificare che l’utente possa consultare il manuale. & NI\\ \hline
                \rowcolor[HTML]{EFEFEF}
                TA19 & Verificare che l’utente possa salvare la sessione di lavoro in corso o ripristinarne una precedente. & I\\ \hline
                \rowcolor[HTML]{C0C0C0}
                TA20 & Verificare che l'applicazione sia compatibile con il browser \textit{Chrome} dalla versione 61. & I\\ \hline
                \rowcolor[HTML]{EFEFEF}
                TA21 & Verificare che l'applicazione sia compatibile con il browser \textit{Edge} dalla versione 16. & I\\ \hline
                \rowcolor[HTML]{C0C0C0}
                TA22 & Verificare che l'applicazione sia compatibile con il browser \textit{Firefox} dalla versione 60. & I\\ \hline
                \rowcolor[HTML]{EFEFEF}
                TA23 & Verificare che l'applicazione sia compatibile con il browser \textit{Opera} dalla versione 48. & I\\ \hline
                \rowcolor[HTML]{C0C0C0}
                TA24 & Verificare che l'applicazione sia compatibile con il browser \textit{Safari} dalla versione 10.1. & I\\ \hline
                \caption{Tabella dei test di accettazione}
            \end{longtable}
        \end{center}

\section{Specifica dei test}
\begin{center}
    \renewcommand\arraystretch{1.5}
    \centering
    \begin{longtable}{|p{1.5cm}|p{11cm}|p{1cm}|}
    \hline
    \rowcolor[HTML]{036400}
    \textcolor{white}{\textbf{Codice}} & \textcolor{white}{\textbf{Descrizione}} & \textcolor{white}{\textbf{Stato}} \\ \hline
        \rowcolor[HTML]{EFEFEF}
        STF1 & L'utente deve poter caricare dati nel sistema tramite file CSV$_G$ oppure caricare una sessione precedente tramite un file JSON$_G$. \`E necessario verificare che: \begin{itemize}
            \item L'utente possa caricare i dati mediante gli appositi bottoni;
            \item I file caricati siano sintatticamente corretti;
            \item Sia visualizzato a schermo un messaggio con l'esito dell'operazione.
        \end{itemize} & I\\ \hline
        \rowcolor[HTML]{C0C0C0}
        STF2 & L'utente deve essere informato in caso di errore nel caricamento dei file. \'E necessario verificare che: \begin{itemize}
            \item L'avviso di errore sia chiaro e ben visibile;
            \item Sia possibile inserire nuovamente i file.
        \end{itemize} & I\\ \hline
        \rowcolor[HTML]{EFEFEF}
        STF3.1 & L'applicazione deve fornire la scelta del grafico. Verificare che: \begin{itemize}
            \item Sia disponibile la scelta del grafico \textit{Scatter Plot}$_G$;
            \item Sia disponibile la scelta del grafico \textit{Parallel Coordinates}$_G$;
            \item Sia disponibile la scelta del grafico \textit{Force-Direct Graph}$_G$;
            \item Sia disponibile la scelta del grafico \textit{Sankey Diagram}$_G$;
            \item La scelta dell'utente venga elaborata correttamente dall'applicazione.
        \end{itemize} & I\\ \hline
        \rowcolor[HTML]{C0C0C0}
        STF3.2 & L'applicazione deve fornire la configurazione del grafico \textit{Scatter Plot}. Verificare che: \begin{itemize}
            \item L'utente possa scegliere quali combinazioni di dimensioni l'applicativo deve visualizzare.
        \end{itemize} & I\\ \hline
        \rowcolor[HTML]{EFEFEF}
        STF3.3 & L'applicazione deve fornire la configurazione del grafico \textit{Parallel Coordinates}. Verificare che: \begin{itemize}
            \item L'utente possa scegliere quali combinazioni di dimensioni l'applicativo deve visualizzare.
        \end{itemize} & I\\ \hline
        \rowcolor[HTML]{C0C0C0}
        STF3.4 & L'applicazione deve fornire la configurazione del grafico \textit{Force-Direct Graph}. Verificare che: \begin{itemize}
            \item L'utente possa scegliere quali combinazioni di dimensioni l'applicativo deve visualizzare.
        \end{itemize} & I\\ \hline
        \rowcolor[HTML]{EFEFEF}
        STF3.5 & L'applicazione deve fornire la configurazione del grafico \textit{Sankey Diagram}. Verificare che: \begin{itemize}
            \item L'utente possa scegliere quali combinazioni di dimensioni l'applicativo deve visualizzare.
        \end{itemize} & I\\ \hline
        \rowcolor[HTML]{C0C0C0}
        STF4.1 & L'applicazione deve permettere di modificare il grafico \textit{Scatter Plot}. Verificare che: \begin{itemize}
            \item Sia possibile modificare la visualizzazione dei punti;
            \item Sia possibile modificare i colori;
            \item La rappresentazione grafica sia la più chiara possibile;
            \item Le modifiche vengano visualizzate correttamente.
        \end{itemize} & NI\\ \hline
        \rowcolor[HTML]{EFEFEF}
        STF4.2 & L'applicazione deve permettere di modificare il grafico \textit{Parallel Coordinates}. Verificare che: \begin{itemize}
            \item Sia possibile modificare l'opacità delle curve del grafico;
            \item Sia possibile modificare la curvatura delle linee del grafico;
            \item Sia possibile modificare la forza di raggruppamento delle linee del grafico;
            \item La rappresentazione grafica sia la più chiara possibile;
            \item Le modifiche vengano visualizzate correttamente.
        \end{itemize} & NI\\ \hline
        \rowcolor[HTML]{C0C0C0}
        STF4.3 & L'applicazione deve permettere di modificare il grafico \textit{Force-Direct Graph}. Verificare che: \begin{itemize}
            \item Sia possibile modificare l'intensità della forza di repulsione del grafico;
            \item Sia possibile modificare l'intensità della forza di tensione del grafico;
            \item Sia possibile modificare i colori al grafico;
            \item La rappresentazione grafica sia la più chiara possibile;
            \item Le modifiche vengano visualizzate correttamente.
        \end{itemize} & NI\\ \hline
        \rowcolor[HTML]{EFEFEF}
        STF4.4 & L'applicazione deve permettere di modificare il grafico \textit{Sankey Diagram}. Verificare che: \begin{itemize}
            \item Sia possibile modificare i colori dei link del grafico;
            \item Sia possibile modificare l'opacità dei link del grafico;
            \item Sia possibile modificare l'allineamento dei nodi del grafico ;
            \item La rappresentazione grafica sia la più chiara possibile;    
            \item Le modifiche vengano visualizzate correttamente.
        \end{itemize} & NI\\ \hline
        \rowcolor[HTML]{C0C0C0}
        STF5 & L'applicazione deve permettere di impostare vari filtri. Verificare che: \begin{itemize}
            \item Sia possibile impostare un filtro sui dati;
            \item Sia possibile impostare un filtro sugli utenti;
            \item Sia possibile impostare un filtro sul tipo di evento;
            \item Sia possibile impostare un filtro sulla data;
            \item Sia possibile impostare un filtro sul tipo di applicazione;
            \item Il grafico rispetti le personalizzazioni scelte;
            \item Sia possibile modificare i parametri scelti;
            \item Le modifiche vengano visualizzate correttamente.
        \end{itemize} & I\\ \hline
        \rowcolor[HTML]{EFEFEF}
        STF6 & L'applicazione deve permettere la visualizzazione in qualunque momento del manuale utente. Verificare che: \begin{itemize}
            \item L'utente possa individuare e consultare facilmente la guida;
            \item La guida sia comprensibile a qualunque tipologia di utente;
            \item La guida descriva tutti gli utilizzi e le caratteristiche dell'applicazione.
        \end{itemize} & NI\\ \hline
        \rowcolor[HTML]{C0C0C0}
        STF7 & L'applicazione deve permettere il salvataggio della configurazione$_G$ e delle impostazioni selezionate per permettere di ripristinarle in un momento successivo. Verificare che: \begin{itemize}
            \item Sia disponibile un bottone per il salvataggio della sessione;
            \item La sessione venga salvata correttamente;
        \end{itemize}& I\\ \hline
        \caption{Tabella della specifica dei test}
    \end{longtable}
\end{center}

    Attualmente lo stato di tutti i test è contrassegnato dalla sigla \textit{NI} che indica "Non Implementato".
\begin{center}
    \renewcommand\arraystretch{1.5}
    \centering
    \begin{longtable}{|p{3cm}|p{10cm}|} \hline
        \rowcolor[HTML]{036400}
        \textcolor{white}{\textbf{Codice Test}} & \textcolor{white}{\textbf{Codice Requisiti}} \\ \hline
        \rowcolor[HTML]{EFEFEF}
        STF1 & RF.1.1 RF.2.3\\ \hline
        \rowcolor[HTML]{C0C0C0}
        STF2 & RF.1.2 RF.1.4 \\ \hline
        \rowcolor[HTML]{EFEFEF}
        STF3.1 & RF.1.5.1 RF.1.5.2 RF.1.5.3 RF.1.5.4 \\ \hline
        \rowcolor[HTML]{C0C0C0}
        STF3.2 & RF.1.5.5 \\ \hline
        \rowcolor[HTML]{EFEFEF}
        STF3.3 & RF.1.5.6 \\ \hline
        \rowcolor[HTML]{C0C0C0}
        STF3.4 & RF.1.5.7 \\ \hline
        \rowcolor[HTML]{EFEFEF}
        STF3.5 & RF.1.5.8 \\ \hline
        \rowcolor[HTML]{C0C0C0}
        STF4.1 & RF.2.6.1.1 RF.2.6.1.2\\ \hline
        \rowcolor[HTML]{EFEFEF}
        STF4.2 & RF.2.6.2.1 RF.2.6.2.2 RF.2.6.2.3\\ \hline
        \rowcolor[HTML]{C0C0C0}
        STF4.3 & RF.2.6.3.1 RF.2.6.3.2 RF.2.6.3.3\\ \hline
        \rowcolor[HTML]{EFEFEF}
        STF4.4 & RF.2.6.4.1 RF.2.6.4.2 RF.2.6.4.3\\ \hline
        \rowcolor[HTML]{C0C0C0}
        STF5 & RF.2.7 RF.2.7.1 RF.2.7.2 RF.2.7.3 RF.2.7.4 RF.2.7.5 \\ \hline
        \rowcolor[HTML]{EFEFEF}
        STF6 & RF.2.8 \\ \hline
        \rowcolor[HTML]{C0C0C0}
        STF7 & RF.2.9 \\ \hline
        \caption{Tracciamento test - requisiti funzionali}
    \end{longtable}
\end{center}
