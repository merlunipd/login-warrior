\chapter{Valutazioni per il miglioramento}
 In questa sezione il team riporta le criticità riscontrate durante
 lo svolgimento del progetto al fine di migliorare la qualità del
 lavoro svolto.

\section{Valutazione sull'organizzazione}
\begin{center}
    \renewcommand{\arraystretch}{1.25}
    \begin{longtable}{|p{0.16\linewidth}|p{0.11\linewidth}|p{0.3\linewidth}|p{0.3\linewidth}|}
    \hline
    \rowcolor[HTML]{036400}
    {\color[HTML]{EFEFEF} \textbf{Problema}} & {\color[HTML]{EFEFEF} \textbf{Gravità}} & {\color[HTML]{EFEFEF} \textbf{Descrizione}} & {\color[HTML]{EFEFEF} \textbf{Soluzione}} \\ \hline
    \rowcolor[HTML]{EFEFEF}
    Suddivisione compiti& Bassa & Inizialmente è stato deciso di suddividere il gruppo in sottogruppi e assegnare a ciascuno di essi un documento differente da redigere. Questa scelta si è rivelata svantaggiosa a causa della dipendenza tra alcuni documenti che impediva il lavoro parallelo dei sottogruppi. & Il gruppo ha deciso di convergere le proprie forze per la realizzazione sequenziale dei documenti con dipendenza. \\ \hline
    \rowcolor[HTML]{C0C0C0}
    Organizzativo & Bassa  & Durante il periodo della seconda milestone$_G$, dal 20/12/21 al 10/01/22, il gruppo si è reso conto di non essere in grado di rispettare le ore preventivate. &Il gruppo ha consuntivato meno ore rispetto a quelle preventivate. Inoltre per evitare che tale errore possa ripetersi si è deciso di porre maggior attenzione nell'attività di previsione oraria, andando a segnalare le ore che si è sicuri verranno impiegate per l'avanzamento delle attività di progetto.  \\ \hline
    \rowcolor[HTML]{EFEFEF}
    Organizzativo & Media & Durante il periodo della terza milestone, dal 15/01/2022 al 04/02/2022, il gruppo si è reso conto di non essere in grado di rispettare le ore preventivate. La presenza della sessione d'esame ha occupato più tempo del previsto e la stima di disponibilità oraria è risultata quindi errata. & Il gruppo ha consuntivato meno ore rispetto a quelle preventivate. Per evitare che questa situazione possa ripresentarsi, visto che è già la seconda volta, il gruppo ha capito che deve prestare ancora più attenzione nell'attività di previsione oraria, cercando di prevedere quali potrebbero essere le problematiche che possono presentarsi durante il periodo.  \\ \hline
    \rowcolor[HTML]{C0C0C0}
    Resoconto attività di verifica & Media & Fino a poco prima della revisione RTB il gruppo non ha controllato il progresso dei documenti utilizzando le metriche che si era posto di sfruttare. In questo modo, in prossimità della revisione, è stato impossibile poter modificare in modo adeguato i documenti per renderli il più conformi possibile a quanto inizialmente prestabilito. & Il gruppo ha deciso di svolgere i dovuti controlli ai documenti ad ogni termine delle milestone, indicativamente ogni due settimane, in modo da avere una visione più chiara del loro sviluppo.\\ \hline
    \hline
    \caption{Tabella della valutazione sull'organizzazione}
    \end{longtable}
    
\end{center}

\section{Valutazione sui ruoli}
\begin{center}
    \renewcommand{\arraystretch}{1.25}
\begin{table}[H]
    \centering
    \begin{longtable}{|p{0.16\linewidth}|p{0.11\linewidth}|p{0.3\linewidth}|p{0.3\linewidth}|}
   \hline
    \rowcolor[HTML]{036400}
    {\color[HTML]{EFEFEF} \textbf{Problema}} & {\color[HTML]{EFEFEF} \textbf{Gravità}} & {\color[HTML]{EFEFEF} \textbf{Descrizione}} & {\color[HTML]{EFEFEF} \textbf{Soluzione}} \\ \hline
    \rowcolor[HTML]{EFEFEF}
    Responsabile & Media & Inizialmente il gruppo ha incontrato svariate difficoltà nella distribuzione delle ore e nella suddivisione equa dei compiti. & Il gruppo ha deciso di puntare a previsioni di più breve durata. \\ \hline
    \rowcolor[HTML]{C0C0C0}
    Verificatore & Bassa & Nelle fasi iniziali del progetto il ruolo è stato svolto con superficialità, causando un incremento degli errori nei documenti. & Il gruppo ha deciso di porre maggior attenzione e tempo nelle attività di verifica. \\ \hline
    \caption{Tabella della valutazione sui ruoli}
    \end{longtable}

\end{table}
\end{center}


\section{Valutazione sugli strumenti di lavoro}
\begin{center}
    \renewcommand{\arraystretch}{1.25}
    \begin{table}[H]
        \centering
        \begin{longtable}{|p{0.16\linewidth}|p{0.11\linewidth}|p{0.3\linewidth}|p{0.3\linewidth}|}
        \hline
        \rowcolor[HTML]{036400}
        {\color[HTML]{EFEFEF} \textbf{Problema}} & {\color[HTML]{EFEFEF} \textbf{Gravità}} & {\color[HTML]{EFEFEF} \textbf{Descrizione}} & {\color[HTML]{EFEFEF} \textbf{Soluzione}} \\ \hline
        \rowcolor[HTML]{EFEFEF}
        GitHub & Bassa & Inizialmente alcuni membri del gruppo hanno riscontrato difficoltà nell'utilizzo dello strumento di versionamento a causa dell'inesperienza. & I membri del gruppo con le lacune hanno svolto un'attività di autoapprendimento utilizzando anche le risorse fornite dai compagni più esperti. \\ \hline
        \rowcolor[HTML]{C0C0C0}
        \LaTeX & Media & L'utilizzo del software si è dimostrata più complessa di quanto ci si aspettasse, in particolare per quanto riguarda il posizionamento delle immagini e delle tabelle. & Il gruppo ha investito una maggior quantità di tempo nelle attività di autoapprendimento per comprendere meglio il funzionamento dello strumento in questione. \\ \hline
        \caption{Tabella dulla valutazione sugli strumenti di lavoro}
        \end{longtable}
    \end{table}
\end{center}