\chapter{Introduzione}
\section{Premessa}

Il \textit{Piano di Qualifica} è un documento su cui si prevede di lavorare per l'intera durata del progetto. Molti contenuti di questo documento sono di natura instabile, come alcune metriche che non sono applicabili nella fase iniziale e che solo con il loro utilizzo pratico si può valutarne l'effettiva utilità. Anche i processi selezionati possono essere soggetti a cambiamenti, dato che possono rivelarsi insufficienti o inadeguati agli scopi del progetto e al modo di lavorare del gruppo.
Per tutte queste ragioni il documento è prodotto in maniera incrementale e suoi contenuti iniziali sono da considerarsi incompleti.

\section{Scopo del documento}
Il \textit{Piano di Qualifica} è un documento che:
\begin{itemize}
    \item Specifica gli obiettivi
    quantitativi di qualità di prodotto e di processo;
    \item Espone le
    metodologie di controllo e le misurazioni di queste qualità tramite
    opportune metriche;
    \item Definisce quanti e quali test eseguire per verificare il corretto funzionamento
    e la qualità dei processi e del prodotto;
    \item Applica questi test e ne documenta l'esito;
    \item Crea un cruscotto$_G$ di supporto che fornisce
    una visione dello stato corrente degli obiettivi.
\end{itemize}

\section{Scopo del prodotto}
Il capitolato proposto dall'azienda \textit{Zucchetti S.p.A} ha come obiettivo quello di creare un'applicazione di visualizzazione di dati con numerose dimensioni che permettono di rintracciare eventuali anomalie a colpo d'occhio. Lo scopo del prodotto è quindi quello di fornire all'utente diversi tipi di visualizzazione di dati in modo da rendere più veloce ed efficace l'individuazione di anomalie.

\section{Glossario}
Al fine di evitare incomprensioni relative alla terminologia usata all'interno del documento, viene fornito un Glossario in grado di dare una definizione precisa per ogni vocabolo potenzialmente ambiguo.

\section{Riferimenti}
\subsection{Riferimenti normativi}
\begin{itemize}
  \item \textit{Norme di Progetto v1.0.0}
\end{itemize}
\subsection{Riferimenti informativi}
\begin{itemize}
  \item \textbf{Capitolato d'appalto C5 - Login Warrior}
          \url{https://www.math.unipd.it/~tullio/IS-1/2021/Progetto/C5.pdf}
  \item \textbf{Qualità di processo}
          \url{https://www.math.unipd.it/~tullio/IS-1/2021/Dispense/T13.pdf}
  \item \textbf{Qualità di prodotto}
          \url{https://www.math.unipd.it/~tullio/IS-1/2021/Dispense/T12.pdf}
  \item \textbf{Verifica e validazione}
          \url{https://www.math.unipd.it/~tullio/IS-1/2021/Dispense/T14.pdf}
          \url{https://www.math.unipd.it/~tullio/IS-1/2021/Dispense/T15.pdf}
          \url{https://www.math.unipd.it/~tullio/IS-1/2021/Dispense/T16.pdf}
  \item \textbf{Ciclo di Deming}
          \url{https://it.wikipedia.org/wiki/Ciclo_di_Deming}
  \item \textbf{Indice di Gulpease}
          \url{https://it.wikipedia.org/wiki/Indice_Gulpease}
\end{itemize}
