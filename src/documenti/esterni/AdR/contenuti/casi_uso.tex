\chapter{Casi d'Uso}

\todo{rivedere la numerazione dei casi d'uso}

\begin{figure}[h]
  \centering
  \includegraphics[width=0.6\textwidth]{UML_general}
  \caption{Inizializzazione del sistema}
\end{figure}
\section{UC1 - Caricamento dataset}
\begin{itemize}
  \item \textbf{Descrizione:} l'utente vuole analizzare un nuovo dataset non presente nel sistema;
  \item \textbf{Attore primario:} utente;
  \item \textbf{Precondizioni:} il sistema è raggiungibile e funzionante. L’utente ha a disposizione un dataset in formato CSV;
  \item \textbf{Postcondizioni:} i dati presenti nel file vengono caricati nel sistema. Viene visualizzato un messaggio che indica il corretto caricamento dei dati;
  \item \textbf{Scenario principale:}
  \begin{enumerate}
    \item L'utente accede al sistema;
    \item L'utente sceglie un file in formato CSV presente in locale e lo carica nel sistema;
    \item L'utente è pronto ad analizzare i dati.
  \end{enumerate}
  \item \textbf{Estensioni:} nel caso in cui il file sia in un formato non valido o i dati non siano validi:
    \begin{enumerate}
      \item Il caricamento non va a buon fine;
      \item Viene visualizzato un errore esplicativo [UC9].
    \end{enumerate}
\end{itemize}

\section{UC2 - Caricamento sessione salvata}
\begin{itemize}
  \item \textbf{Descrizione:} l'utente vuole riprendere ad analizzare da dove si era interrotto
  o ha la necessità di visualizzare una sessione precedente;
  \item \textbf{Attore Primario:} utente;
  \item \textbf{Precondizioni:} l'utente che avvia l'applicativo ha salvato almeno una sessione di lavoro precedente;
  \item \textbf{Postcondizioni:} i dati di una sessione precedentemente salvata vengono ricaricati nel sistema. Viene visualizzato un messaggio che indica il corretto caricamento dei dati;
  \item \textbf{Scenario Principale:}
  \begin{enumerate}
    \item L'utente accede al sistema;
    \item L'utente sceglie la sessione da caricare selezionando il file JSON desiderato tra quelli disponibili,
    cioè tra le sessioni salvate in precedenza;
    \item L'utente riprende da dove aveva salvato.
  \end{enumerate}
  \item \textbf{Estensioni:} nel caso in cui il file JSON selezionato non è leggibile per qualche possibile errore di salvataggio:
    \begin{enumerate}
      \item Fallisce il caricamento della sessione precedente;
      \item Viene visualizzato un errore esplicativo [UC9].
    \end{enumerate}
\end{itemize}


\section{UC3 - Selezione dimensioni}
 \begin{itemize}
     \item \textbf{Descrizione:} selezione dimensioni con cui verrà visualizzato il grafico;
     \item \textbf{Attore primario:} utente;
     \item \textbf{Precondizioni:} il sistema è stato inizializzato [UC1];
     \item \textbf{Postcondizioni:} le dimensioni vengono aggiornate nel sistema;
     \item \textbf{Scenario principale:}
     \begin{enumerate}
         \item vengono mostrate all'utente le dimensioni di default e altre dimensioni tra cui scegliere;
         \item l'utente seleziona la/e dimensione/i che più ritiene utile/i.
     \end{enumerate}
 \end{itemize}


\section{UC4 - Selezione tipo di grafico}
\begin{figure}[H]
 \includegraphics[width=\textwidth]{uc4.png}
 \vspace{-5mm}
 \caption*{Figura 2: UC4 - Selezione tipo di grafico}
\end{figure}

 \begin{itemize}
     \item \textbf{Descrizione:} viene visualizzata la scelta della tipologia di grafico;
     \item \textbf{Attore primario:} utente;
     \item \textbf{Precondizioni:} il sistema è stato inizializzato [UC1];
     \item \textbf{Postcondizioni:} viene visualizzato il grafico desiderato;
     \item \textbf{Scenario principale:} l'utente sceglie la visualizzazione più consona tra quelle disponibili;
     \item \textbf{Generalizzazioni:} l'utente può selezionare una tra le possibili opzioni:
     \begin{enumerate}
         \item \textit{Scatter Plot} [UC4.1];
         \item \textit{Parallel Coordinates} [UC4.2];
         \item \textit{Force Directed Graph} [UC4.3];
         \item \textit{Diagramma di Sankey} [UC4.4].
     \end{enumerate}
 \end{itemize}


\section{UC4 - Personalizzazione Visualizzazione}

\begin{figure}[h]
  \centering
  \includegraphics[width=0.6\textwidth]{ucX.png}
  \caption{UC4 - Personalizzazione Visualizzazione}
\end{figure}

\begin{itemize}
  \item \textbf{Descrizione}: l'utente ha la possibilità di modificare vari aspetti visivi del grafico;
  \item \textbf{Attore primario}: utente;
  \item \textbf{Precondizioni}: l'utente ha selezionato le dimensioni del grafico [UC3] e l'applicativo lo ha generato;
  \item \textbf{Postcondizioni}: le modifiche apportate al grafico vengono visualizzate;
  \item \textbf{Scenario principale}:
  \begin{itemize}
    \item L'utente può scegliere le caratteristiche da modificare tra:
      \begin{itemize}
        \item Zoom [UC4.1];
        \item Filtri sui dati [UC4.2];
        \item Colori [UC4.3];
        \item Scala degli assi [UC4.4];
        \item Attivare/Disattivare le animazioni [UC4.5];
      \end{itemize}
    \item Il grafico viene visualizzato con le nuove caratteristiche;
  \end{itemize}
  \item \textbf{Estensioni}:
    \begin{itemize}
      \item L'utente inserisce dei filtri non validi [UC 4 + 1];
      \item L'utente inserisce una scala degli assi non valida [UC4 + 2].
    \end{itemize}
\end{itemize}

\section{UC4 + 1 - Errore scelta filtri}
\begin{itemize}
  \item \textbf{Descrizione}: l'utente sceglie dei filtri non validi;
  \item \textbf{Attore primario}: utente;
  \item \textbf{Precondizioni}: l'utente sceglie dei filtri che non permettono una corretta visualizzazione del grafico;
  \item \textbf{Postcondizioni}: l'utente visualizza un messaggio di errore;
  \item \textbf{Scenario principale}:
    \begin{itemize}
      \item L'utente visualizza un messaggio di errore esplicativo;
      \item L'utente clicca \texttt{Capito} per tornare alla visualizzazione del grafico con la personalizzazione di default.
    \end{itemize}
\end{itemize}

\section{UC4 + 2 - Errore scala degli assi}
\begin{itemize}
  \item \textbf{Descrizione}: l'utente scelglie una scala non valida;
  \item \textbf{Attore primario}: utente;
  \item \textbf{Precondizioni}: l'utente sceglie una scala degli assi che non permette una corretta visualizzazione del grafico;
  \item \textbf{Postcondizioni}: l'utente visualizza un messaggio di errore;
  \item \textbf{Scenario principale}:
    \begin{itemize}
      \item L'utente visualizza un messaggio di errore esplicativo;
      \item L'utente clicca \texttt{Capito} per tornare alla visualizzazione del grafico con la personalizzazione di default.
    \end{itemize}
\end{itemize}

\section{Accesso ai manuali}
\begin{figure}[h]
  \centering
  \includegraphics[width=0.6\textwidth]{UC_manuali}
  \caption{UC6 UC7 - Accesso ai manuali utente e sviluppatore}
\end{figure}
\subsection{UC6 - Accesso al Manuale Utente}

\begin{itemize}
  \item \textbf{Descrizione}: l'utente che ha un dubbio o vuole più informazioni sull'utilizzo dell'applicazione, deve avere accesso rapido al manuale utente;
  \item \textbf{Attore primario}: utente;
  \item \textbf{Precondizioni}: nessuna, l'opzione di accesso ai manuali deve essere sempre disponibile all'utente;
  \item \textbf{Postcondizioni}: viene visualizzato il manuale utente;
  \item \textbf{Scenario principale}:
  \begin{enumerate}
    \item L'utente seleziona il "manuale utente";
    \item Viene visualizzato il manuale utente.
  \end{enumerate}
\end{itemize}

\subsection{UC7 - Accesso al Manuale Sviluppatore}

\begin{itemize}
  \item \textbf{Descrizione}: essendo Login Warrior un progetto open source, un qualsiasi sviluppatore deve avere accesso ad un manuale sviluppatore (sia per manutenzione, sia per estendere il software);
  \item \textbf{Attore primario}: sviluppatore;
  \item \textbf{Precondizioni}: nessuna, l'opzione di accesso ai manuali deve essere sempre disponibile allo sviluppatore;
  \item \textbf{Postcondizioni}: viene visualizzato il manuale sviluppatore;
  \item \textbf{Scenario principale}:
  \begin{enumerate}
    \item Lo sviluppatore seleziona il "manuale sviluppatore";
    \item Viene visualizzato il manuale sviluppatore.
  \end{enumerate}
\end{itemize}

\section{UC6 - Salvataggio Sessione}

\begin{itemize}
  \item \textbf{Descrizione:} l'utente salva la sessione di lavoro;
  \item \textbf{Attore primario:} utente;
  \item \textbf{Precondizioni:} l'utente ha svolto una sessione di lavoro sull'applicazione, in particolare potrebbe aver scelto un grafico specifico, impostato le dimensioni volute e modificato i parametri personalizzando la visualizzazione;
  \item \textbf{Postcondizioni:} l'utente possiede un file JSON in grado di recuperare grafico, dimensioni e parametri impostati durante la sessione di lavoro;
  \item \textbf{Scenario principale:}
  \begin{enumerate}
    \item L'utente sta lavorando sull'applicazione;
    \item L'utente seleziona la funzionalità ``Salvataggio Sessione'';
    \item L'utente seleziona la directory in cui salvare il file JSON.
  \end{enumerate}
\end{itemize}

\section{UC9- Gestione errore}
\begin{itemize}
  \item \textbf{Attore Primario:} utente;
  \item \textbf{Precondizioni:} l'utente compie un'azione che fa fallire il corretto funzionamento dell'applicazione;
  \item \textbf{Postcondizioni:} l'utente visualizza un messaggio di errore e l'azione non va a buon fine.
  \item \textbf{Scenario Principale:}
  \begin{enumerate}
    \item L'utente visualizza un messaggio di errore esplicativo;
  \end{enumerate}
\end{itemize}
