\chapter{Requisiti}

\renewcommand\arraystretch{1,5}

\section{Introduzione}
Il gruppo \textit{MERL} dopo un'attenta analisi ha individuato i seguenti requisiti che il prodotto finale andrà a soddisfare. Questi sono organizzati in forma tabellare e la loro struttura segue le regole definite nel documento \textit{Norme di Progetto}.

\section{Requisiti Funzionali}

\begin{center}
  \centering
  \begin{longtable}{|L{2cm}|C{5,5cm}|C{3cm}|C{2cm}|}
    \hline
    \rowcolor[HTML]{036400}
    \textcolor[HTML]{FFFFFF}{\textbf{Codice}} & \textcolor[HTML]{FFFFFF}{\textbf{Descrizione}} & \textcolor[HTML]{FFFFFF}{\textbf{Classificazione}} & \textcolor[HTML]{FFFFFF}{\textbf{Fonti}}
    \\ \hline
    \rowcolor[HTML]{EFEFEF}
    RF.1.1 & L'utente deve poter caricare i dati tramite un nuovo dataset & Obbligatorio & Capitolato - UC1 \\ \hline
    \rowcolor[HTML]{C0C0C0}
    RF.1.2 & Visualizzazione messaggio di errore in caso di problemi durante il caricamento dati & Obbligatorio & UC2 \\ \hline
    \rowcolor[HTML]{EFEFEF}
    RF.2.3 & L'utente deve poter caricare una sessione precedentemente salvata & Desiderabile & UC3 \\ \hline
    \rowcolor[HTML]{C0C0C0}
    RF.1.4 & Visualizzazione messaggio di errore in caso di problemi durante il caricamento della sessione & Obbligatorio & UC4 \\ \hline
    \rowcolor[HTML]{EFEFEF}
    RF.1.5 & L'utente deve poter selezionare il grafico da visualizzare & Obbligatorio & Capitolato - UC5 \\ \hline
    \rowcolor[HTML]{C0C0C0}
    RF.1.5.1 & L'utente deve poter selezionare il grafico \textit{Scatter Plot} & Obbligatorio & Capitolato - UC5.1 \\ \hline
    \rowcolor[HTML]{EFEFEF}
    RF.1.5.2 & L'utente deve poter selezionare il grafico \textit{Parallel Coordinates} & Obbligatorio & Capitolato - UC5.2 \\ \hline
    \rowcolor[HTML]{C0C0C0}
    RF.1.5.3 & L'utente deve poter selezionare il grafico \textit{Force-Direct Graph} & Obbligatorio & Capitolato - UC5.3 \\ \hline
    \rowcolor[HTML]{EFEFEF}
    RF.1.5.4 & L'utente deve poter selezionare il grafico \textit{Sankey Diagram} & Obbligatorio & Capitolato - UC5.4 \\ \hline
    \rowcolor[HTML]{C0C0C0}
    RF.1.? & L'utente deve poter personalizzare la visualizzazione scelta & Obbligatorio & UC? \\ \hline
    \rowcolor[HTML]{EFEFEF}
    RF.1.?.1.1 & L'utente deve poter modificare la visualizzazione dei punti & Obbligatorio & UC?.1 \\ \hline
    \rowcolor[HTML]{C0C0C0}
    RF.1.?.1.2 & L'utente deve poter scegliere i colori & Obbligatorio & UC?.1 \\ \hline
    \rowcolor[HTML]{EFEFEF}
    RF.2.?.3.1 & L'utente deve poter scegliere la funzione di forza & Desiderabile & UC?.3 \\ \hline
    \rowcolor[HTML]{C0C0C0}
    RF.1.?.3.2 & L'utente deve poter scegliere l'intensità delle forze di tensione e repulsione & Obbligatorio & UC?.3 \\ \hline
    \rowcolor[HTML]{EFEFEF}
    RF.1.?.3.3 & L'utente deve poter scegliere gli stili per la visualizzazione del grafico & Obbligatorio & UC?.3 \\ \hline
    \rowcolor[HTML]{C0C0C0}
    RF.2.?.4.1 & L'utente deve poter scegliere la colorazione dei link & Obbligatorio & UC?.4 \\ \hline
    \rowcolor[HTML]{EFEFEF}
    RF.1.?.4.2 & L'utente deve poter impostare l'opacità dei link & Obbligatorio & UC?.4 \\ \hline
    \rowcolor[HTML]{C0C0C0}
    RF.2.?.4.3 & L'utente deve poter scegliere l'allinamento dei link & Desiderabile & UC?.4 \\ \hline
    \rowcolor[HTML]{EFEFEF}
    RF.1.7 & Visualizzazione messaggio di errore in caso di filtri scelti in modo scorretto & Obbligatorio & UC7 \\ \hline
    \rowcolor[HTML]{C0C0C0}
    RF.2.8 & L'utente deve poter accedere al manuale utente & Desiderabile & UC8 \\ \hline
    \rowcolor[HTML]{EFEFEF}
    RF.2.9 & L'utente deve poter salvare la sessione in corso & Desiderabile & UC9 \\ \hline

    \caption{Tabella dei requisiti funzionali}
  \end{longtable}
\end{center}

\section{Requisiti di Qualità}
\begin{center}
  \centering
  \begin{longtable}{|L{1,5cm}|C{5,5cm}|C{3cm}|C{2cm}|}
    \hline
    \rowcolor[HTML]{036400}
    \textcolor[HTML]{FFFFFF}{\textbf{Codice}} & \textcolor[HTML]{FFFFFF}{\textbf{Descrizione}} & \textcolor[HTML]{FFFFFF}{\textbf{Classificazione}} & \textcolor[HTML]{FFFFFF}{\textbf{Fonti}}
    \\ \hline
    \rowcolor[HTML]{EFEFEF}
    RQ.1.1 & Deve essere fornito un manuale utente per l'utilizzo & Obbligatorio & Capitolato \\ \hline
    \rowcolor[HTML]{C0C0C0}
    RQ.1.2 & Il prodotto deve essere open source & Obbligatorio & Capitolato \\ \hline
    \rowcolor[HTML]{EFEFEF}
    RQ.1.3 & Il codice sorgente deve essere presente su una repository in \textit{GitHub} o in altri repository pubblici & Obbligatorio & Capitolato \\ \hline
    \rowcolor[HTML]{C0C0C0}
    RQ.1.4 & Il prodotto deve essere sviluppato seguendo le \textit{Norme di Progetto} & Obbligatorio & \textit{Norme di Progetto} \\ \hline

    \caption{Tabella dei requisiti di qualità}
  \end{longtable}
\end{center}

\section{Requisiti di Vincolo}
\begin{center}
  \centering
  \begin{longtable}{|L{1,5cm}|C{5,5cm}|C{3cm}|C{2cm}|}
    \hline
    \rowcolor[HTML]{036400}
    \textcolor[HTML]{FFFFFF}{\textbf{Codice}} & \textcolor[HTML]{FFFFFF}{\textbf{Descrizione}} & \textcolor[HTML]{FFFFFF}{\textbf{Classificazione}} & \textcolor[HTML]{FFFFFF}{\textbf{Fonti}}
    \\ \hline
    \rowcolor[HTML]{EFEFEF}
    RV.1.1 & L'interfaccia grafica deve essere sviluppata in \textit{HTML}/\textit{CSS} & Obbligatorio & Capitolato \\ \hline
    \rowcolor[HTML]{C0C0C0}
    RV.1.2 & I grafici devono essere realizzati tramite l'utilizzo di \textit{Javascript} & Obbligatorio & Capitolato \\ \hline
    \rowcolor[HTML]{EFEFEF}
    RV.1.3 & Il prodotto finale deve essere in grado di analizzare file \textit{CSV} & Obbligatorio & Capitolato \\ \hline
    \rowcolor[HTML]{C0C0C0}
    RV.3.4 & Deve essere utilizzata la libreria \textit{D3.js} & Opzionale & Capitolato \\ \hline
    \rowcolor[HTML]{EFEFEF}
    RV.1.5 & Il prodotto finale è supportato dai seguenti browser: \textit{Chrome 61}, \textit{Edge 16}, \textit{Firefox 60}, \textit{Opera 48}, \textit{Safari 10.1} & Obbligatorio & \textit{Javascript Modules} \\ \hline


    \caption{Tabella dei requisiti di vincolo}
  \end{longtable}
\end{center}

\section{Requisiti Prestazionali}
Il gruppo \textit{MERL} non ha individuato alcun requisito prestazionale durante l'analisi del capitolato e delle richieste del proponente.


\section{Tracciamento}

\subsection{Fonte - Requisiti}
\begin{center}
  \centering
  \begin{longtable}{|C{6cm}|C{6cm}|}
    \hline
    \rowcolor[HTML]{036400}
    \textcolor[HTML]{FFFFFF}{\textbf{Fonte}} & \textcolor[HTML]{FFFFFF}{\textbf{Requisiti}} \\ \hline
    \rowcolor[HTML]{EFEFEF}
    Capitolato & RF.1.1 - RF.1.5 - RF.1.5.1 - RF.1.5.2 - RF.1.5.3 - RF.1.5.4 - RQ.1.1 - RQ.1.2 - RQ.1.3 - RV.1.1 - RV.1.2 - RV.1.3 - RV.3.4 \\ \hline
    \rowcolor[HTML]{C0C0C0}
    UC1 & RF.1.1 \\ \hline
    \rowcolor[HTML]{EFEFEF}
    UC2 & RF.1.2 \\ \hline
    \rowcolor[HTML]{C0C0C0}
    UC3 & RF.2.3 \\ \hline
    \rowcolor[HTML]{EFEFEF}
    UC4 & RF.1.4 \\ \hline
    \rowcolor[HTML]{C0C0C0}
    UC5 & RF.1.5 \\ \hline
    \rowcolor[HTML]{EFEFEF}
    UC5.1 & RF.1.5.1 \\ \hline
    \rowcolor[HTML]{C0C0C0}
    UC5.2 & RF.1.5.2 \\ \hline
    \rowcolor[HTML]{EFEFEF}
    UC5.3 & RF.1.5.3 \\ \hline
    \rowcolor[HTML]{C0C0C0}
    UC5.4 & RF.1.5.4 \\ \hline
    \rowcolor[HTML]{EFEFEF}
    UC? & RF.1.? \\ \hline
    \rowcolor[HTML]{C0C0C0}
    UC?.1 & RF.1.?.1.1 RF.1.?.1.2 \\ \hline
    \rowcolor[HTML]{EFEFEF}
    UC?.3 & RF.2.?.3.1 RF.1.?.3.2 RF.1.?.3.3  \\ \hline
    \rowcolor[HTML]{C0C0C0}
    UC?.4 & RF.1.?.4.1 RF.1.?.4.2 RF.2.?.4.3  \\ \hline
    \rowcolor[HTML]{EFEFEF}
    UC7 & RF.1.7 \\ \hline
    \rowcolor[HTML]{C0C0C0}
    UC8 & RF.2.8 \\ \hline
    \rowcolor[HTML]{EFEFEF}
    UC9 & RF.2.9 \\ \hline
    \rowcolor[HTML]{C0C0C0}
    \textit{Norme di Progetto} & RQ.1.4 \\ \hline
    \rowcolor[HTML]{EFEFEF}
    R.V.1.2 & RV.1.5 \\ \hline

    \caption{Tabella di tracciamento fonte-requisiti}
  \end{longtable}
\end{center}

\subsection{Requisito - Fonti}
\begin{center}
  \centering
  \begin{longtable}{|C{6cm}|C{6cm}|}
    \hline
    \rowcolor[HTML]{036400}
    \textcolor[HTML]{FFFFFF}{\textbf{Requisito}} & \textcolor[HTML]{FFFFFF}{\textbf{Fonti}} \\ \hline
    \rowcolor[HTML]{EFEFEF}
    RF.1.1 & Capitolato - UC1 \\ \hline
    \rowcolor[HTML]{C0C0C0}
    RF.1.2 & UC2 \\ \hline
    \rowcolor[HTML]{EFEFEF}
    RF.2.3 & UC3 \\ \hline
    \rowcolor[HTML]{C0C0C0}
    RF.1.4 & UC4 \\ \hline
    \rowcolor[HTML]{EFEFEF}
    RF.1.5 & Capitolato - UC5 \\ \hline
    \rowcolor[HTML]{C0C0C0}
    RF.1.5.1 & Capitolato - UC5.1 \\ \hline
    \rowcolor[HTML]{EFEFEF}
    RF.1.5.2 & Capitolato - UC5.2 \\ \hline
    \rowcolor[HTML]{C0C0C0}
    RF.1.5.3 & Capitolato - UC5.3 \\ \hline
    \rowcolor[HTML]{EFEFEF}
    RF.1.5.4 & Capitolato - UC5.4 \\ \hline
    \rowcolor[HTML]{C0C0C0}
    RF.1.? & UC? \\ \hline
    \rowcolor[HTML]{EFEFEF}
    RF.1.?.1.1 & UC?.1 \\ \hline
    \rowcolor[HTML]{C0C0C0}
    RF.1.?.1.2 & UC?.1 \\ \hline
    \rowcolor[HTML]{EFEFEF}
    RF.2.?.3.1 & UC?.3 \\ \hline
    \rowcolor[HTML]{C0C0C0}
    RF.1.?.3.2 & UC?.3 \\ \hline
    \rowcolor[HTML]{EFEFEF}
    RF.1.?.3.3 & UC?.3 \\ \hline
    \rowcolor[HTML]{C0C0C0}
    RF.1.?.4.1 & UC?.4 \\ \hline
    \rowcolor[HTML]{EFEFEF}
    RF.1.?.4.2 & UC?.4 \\ \hline
    \rowcolor[HTML]{C0C0C0}
    RF.2.?.4.3 & UC?.4 \\ \hline
    \rowcolor[HTML]{EFEFEF}
    RF.1.7 & UC7 \\ \hline
    \rowcolor[HTML]{C0C0C0}
    RF.2.8 & UC9 \\ \hline
    \rowcolor[HTML]{EFEFEF}
    RF.2.9 & UC10 \\ \hline
    \rowcolor[HTML]{C0C0C0}
    RQ.1.1 & Capitolato \\ \hline
    \rowcolor[HTML]{EFEFEF}
    RQ.1.2 & Capitolato \\ \hline
    \rowcolor[HTML]{C0C0C0}
    RQ.1.3 & Capitolato \\ \hline
    \rowcolor[HTML]{EFEFEF}
    RQ.1.4 & \textit{Norme di Progetto} \\ \hline
    \rowcolor[HTML]{C0C0C0}
    RV.1.1 & Capitolato \\ \hline
    \rowcolor[HTML]{EFEFEF}
    RV.1.2 & Capitolato \\ \hline
    \rowcolor[HTML]{C0C0C0}
    RV.1.3 & Capitolato \\ \hline
    \rowcolor[HTML]{EFEFEF}
    RV.3.4 & Capitolato \\ \hline
    \rowcolor[HTML]{C0C0C0}
    RV.1.5 & RV.1.2 \\ \hline

    \caption{Tabella di tracciamento requisito-fonti}
  \end{longtable}
\end{center}


\section{Riepilogo}

\begin{center}
  \centering
  \begin{longtable}{|c|c|c|c|c|}
    \hline
    \rowcolor[HTML]{036400}
    {\color[HTML]{FFFFFF} \textbf{Tipologia}} & {\color[HTML]{FFFFFF} \textbf{Obbligatorio}} & {\color[HTML]{FFFFFF} \textbf{Desiderabile}} & {\color[HTML]{FFFFFF} \textbf{Opzionale}}  & {\color[HTML]{FFFFFF} \textbf{Totale}} \\ \hline
    \rowcolor[HTML]{EFEFEF}
    Funzionale & 16 & 5 & 0 & 21 \\ \hline
    \rowcolor[HTML]{C0C0C0}
    Di Qualità & 4 & 0 & 0 & 4 \\ \hline
    \rowcolor[HTML]{EFEFEF}
    Di Vincolo & 4 & 0 & 1 & 5 \\ \hline
    \rowcolor[HTML]{C0C0C0}
    Prestazionale & 0 & 0 & 0 & 0 \\ \hline

    \caption{Tabella del riepilogo totale}
  \end{longtable}
\end{center}
