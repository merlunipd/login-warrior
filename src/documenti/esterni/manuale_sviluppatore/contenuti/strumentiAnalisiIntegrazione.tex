\chapter{Strumenti per l’analisi e l’integrazione del codice}
\begin{longtable}{|p{5cm}|p{2cm}|p{7cm}|}
    \hline
    \rowcolor[HTML]{036400}
    \textcolor{white}{\textbf{Strumento}} & \textcolor{white}{\textbf{Versione}} & \textcolor{white}{\textbf{Descrizione}} \\ \hline
    \rowcolor[HTML]{059900}
    \multicolumn{3}{|c|}{\textcolor{white}{\textbf{Analisi statica}}} \\
    \rowcolor[HTML]{EFEFEF}
    ESLint & 8.9.0 & \'E uno strumento di analisi statica del codice, viene utilizzato per identificare pattern problematici all'interno di codice JavaScript. Compie sia dei check sulla qualità del codice, sia verifica l'aderenza a un particolare coding style. Il gruppo ha deciso di aderire al coding style definito da AirBnB. \\ \hline
    \rowcolor[HTML]{059900}
    \multicolumn{3}{|c|}{\textcolor{white}{\textbf{Analisi dinamica}}} \\
    \rowcolor[HTML]{C0C0C0}
    Jest & 27.5.1 & \'E un framework di testing per JavaScript, permette di eseguire dei test automatici, definiti dall’utente, per controllare il corretto funzionamento di un programma o progetto. \\ \hline
    \rowcolor[HTML]{059900}
    \multicolumn{3}{|c|}{\textcolor{white}{\textbf{Documentazione}}} \\
    \rowcolor[HTML]{EFEFEF}
    JSDocs & 3.3.10 & \'E un linguaggio di markup che permette di annotare il codice sorgente JavaScript. Comprende uno strumento che permette di generare automaticamente documentazione del codice in formato HTML o RTF. \\ \hline
    \caption{Strumenti per l'analisi e l'integrazine del codice}
\end{longtable}
\renewcommand\arraystretch{1}
