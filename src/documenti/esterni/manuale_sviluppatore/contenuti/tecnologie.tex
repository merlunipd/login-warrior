\chapter{Tecnologie}

\renewcommand\arraystretch{1,5}
{\centering
\begin{longtable}{|C{2.5cm}|C{2cm}|L{9cm}|}
  \hline
  \rowcolor[HTML]{036400}
  \textcolor[HTML]{FFFFFF}{\textbf{Tecnologia}} & \textcolor[HTML]{FFFFFF}{\textbf{Versione}} & \multicolumn{1}{c|}{\textcolor[HTML]{FFFFFF}{\textbf{Descrizione}}} \\ \hline
  \rowcolor[HTML]{036400}
  \multicolumn{3}{|c|}{\textcolor[HTML]{FFFFFF}{\textbf{Linguaggi}}} \\ \hline
  \rowcolor[HTML]{EFEFEF}
  HTML       & 5    & Linguaggio di markup utilizzato per definire gli elementi dell'interfaccia. \\  \hline
  \rowcolor[HTML]{C0C0C0}
  CSS       & 3    & Linguaggio utilizzato per la gestione dello stile degli elementi HTML. \\  \hline
  \rowcolor[HTML]{EFEFEF}
  Javascript       & ES6    & Linguaggio di programmazione ad alto livello, interpretato, multi-paradigma, con tipizzazione debole. Viene utilizzato dal motore del browser per eseguire codice da lato client. Utilizzati i \textit{Moduli ES6} per gestire i file contenenti il codice Javascript. \\  \hline

  \rowcolor[HTML]{036400}
  \multicolumn{3}{|c|}{\textcolor[HTML]{FFFFFF}{\textbf{Librerie}}} \\ \hline
  \rowcolor[HTML]{EFEFEF}
  D3       & 7.4.0    & Libreria Javascript utilizzata per manipolare elementi del DOM$_G$ in base a dati. Permette di creare visualizzazioni e grafici. \\  \hline

  \rowcolor[HTML]{036400}
  \multicolumn{3}{|c|}{\textcolor[HTML]{FFFFFF}{\textbf{Strumenti}}} \\ \hline
  \rowcolor[HTML]{EFEFEF}
  NodeJS       & 17.2.0    & Runtime costruito sul motore V8 di Google per l'esecuzione di codice JavaScript. Utilizzato per accedere a strumenti di supporto allo sviluppo (e.g. JestJS, ESLint) e per la definizione di piccoli script. \\  \hline
  \rowcolor[HTML]{C0C0C0}
  NPM       & 8.1.4    & Package manager per la gestione di dipendenze di progetti NodeJS. \\  \hline
  \rowcolor[HTML]{EFEFEF}
  JestJS       & 27.5    & Strumento per effettuare analisi dinamica di codice Javascript e per generare il code coverage. \\  \hline
  \rowcolor[HTML]{C0C0C0}
  ESLint       & 8.12    & Strumento di analisi statica del codice. Viene utilizzato con le best practices configurate dallo standard \textit{AirBnB}. \\  \hline
  \rowcolor[HTML]{EFEFEF}
  JSDocs       & 3.5.5    & Linguaggio di markup che permette di annotare il codice sorgente Javascript e generare documentazione. \\  \hline
  \rowcolor[HTML]{C0C0C0}
  IndexedDB       & 3.0    & API$_G$ Javascript fornite dai browser per permettere il caching di dati da lato client. \\  \hline
  \rowcolor[HTML]{EFEFEF}
  Git       & 2.34.1    & Strumento di controllo della versione distribuito. Utilizzato per gestire la repository remota su GitHub. \\  \hline
  \caption{Tabella delle tecnologie utilizzate}
\end{longtable}}

\renewcommand\arraystretch{1}
