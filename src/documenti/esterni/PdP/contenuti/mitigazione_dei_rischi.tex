\chapter{Mitigazione dei Rischi}

\section{Rischi legati alle persone}

\begin{table}[H]
    \centering
    \begin{tabular}{|p{2cm}|p{10cm}|}
    \hline
    \multicolumn{2}{|c|}{\textbf{Disponibilità}} \\ \hline
    \multicolumn{1}{|l|}{\textit{Descrizione}} & Il gruppo si è imbattuto in fasi temporali in cui i membri sono stati più o meno attivi in base agli altri impegni universitari. \\ \hline
    \multicolumn{1}{|l|}{\textit{Mitigazione}} & In alcune fasi uno o più membri non sono riusciti a rispettare le precauzioni prese evidenziando alcuni problemi a livello organizzativo, questo si può vedere dalle differenze tra preventivo e consuntivo. Per risolvere questa difficoltà il gruppo ha iniziato a dare maggior importanza alla fase di preventivazione delle ore in modo da essere il più possibile coerenti con le effettive disponibilità dei membri. \\ \hline
    \end{tabular}
\end{table}

\begin{table}[H]
    \centering
    \begin{tabular}{|p{2cm}|p{10cm}|}
    \hline
    \multicolumn{2}{|c|}{\textbf{Problemi interpersonali}} \\ \hline
    \multicolumn{1}{|l|}{\textit{Descrizione}} & All'inizio i membri del gruppo non si conoscevano tra di loro e questo avrebbe potuto provocare incomprensioni interne. \\ \hline
    \multicolumn{1}{|l|}{\textit{Mitigazione}} & Non verificato. Le precauzioni prese sono state sufficienti e di conseguenza non è stato necessario applicare il piano di contingenza. \\ \hline
    \end{tabular}
\end{table}

\begin{table}[H]
    \centering
    \begin{tabular}{|p{2cm}|p{10cm}|}
    \hline
    \multicolumn{2}{|c|}{\textbf{Mancanza di esperienza personale}} \\ \hline
    \multicolumn{1}{|l|}{\textit{Descrizione}} & La poca esperienza dei membri all'interno di un progetto vasto e complesso ha portato ad alcune situazioni di difficoltà. \\ \hline
    \multicolumn{1}{|l|}{\textit{Mitigazione}} & All'interno del gruppo è presente uno spirito di collaborazione che ha permesso aiuti reciproci in situazioni di difficoltà. In questo modo dove un membro si è trovato in difficoltà c'è sempre stato un altro membro pronto a supportarlo al fine di risolvere le difficoltà insieme. Sono state quindi rispettate le precauzioni e, laddove le difficoltà sono risultate maggiori, i membri hanno seguito il piano di contingenza \\ \hline
    \end{tabular}
\end{table}



\section{Rischi legati all'organizzazione}

\begin{table}[H]
    \centering
    \begin{tabular}{|p{2cm}|p{10cm}|}
    \hline
    \multicolumn{2}{|c|}{\textbf{Scarsa pianificazione}} \\ \hline
    \multicolumn{1}{|l|}{\textit{Descrizione}} & La pianificazione di un progetto di queste dimensioni risulta difficile, ancor di più con scarsa esperienza in merito. Si è visto infatti che più di una volta il consuntivo è risultato lontano dal preventivo per quanto riguarda le tempistiche. \\ \hline
    \multicolumn{1}{|l|}{\textit{Mitigazione}} & Come evidenziato in precedenza ci sono state alcune difficoltà a livello di organizzazione. Le precauzioni prese hanno ridotto la gravità dei problemi creati da pianificazioni errate e il piano di contingenza ha fortemente contribuito ad aiutare il gruppo a dare maggior importanza alla fase di preventivazione. \\ \hline
    \end{tabular}
\end{table}



\section{Rischi legati alle tecnologie e agli strumenti}

\begin{table}[H]
    \centering
    \begin{tabular}{|p{2cm}|p{10cm}|}
    \hline
    \multicolumn{2}{|c|}{\textbf{Strumenti sconosciuti}} \\ \hline
    \multicolumn{1}{|l|}{\textit{Descrizione}} & La buona riuscita di un progetto prevede l'utilizzo di strumenti non sempre conosciuti. \\ \hline
    \multicolumn{1}{|l|}{\textit{Mitigazione}} & Le precauzioni prese sono state sufficienti in quanto l'utilizzo di un nuovo strumento è sempre stato anticipato da una discussione e da un'analisi da parte del gruppo. Dopo una prima visione di gruppo ogni membro si è impegnato nello studio individuale dei vari strumenti che sono stati utilizzati. \\ \hline
    \end{tabular}
\end{table}

\begin{table}[H]
    \centering
    \begin{tabular}{|p{2cm}|p{10cm}|}
    \hline
    \multicolumn{2}{|c|}{\textbf{Tecnologie sconosciute}} \\ \hline
    \multicolumn{1}{|l|}{\textit{Descrizione}} & La codifica del software richiede chiaramente la conoscenza di tecnologie specifiche, non sempre conosciute. \\ \hline
    \multicolumn{1}{|l|}{\textit{Mitigazione}} & Le precauzioni prese sono state efficaci in quanto la discussione di gruppo ed il successivo studio individuale hanno portato a buoni risultati. Inoltre è stato organizzato un incontro con un'esperta di \textit{Zucchetti S.p.A.} al fine di migliorare le conoscenze riguardo la libreria \textit{D3.js}. \\ \hline
    \end{tabular}
\end{table}



\section{Rischi legati ai requisiti}

\begin{table}[H]
    \centering
    \begin{tabular}{|p{2cm}|p{10cm}|}
    \hline
    \multicolumn{2}{|c|}{\textbf{Analisi dei requisiti incompleta}} \\ \hline
    \multicolumn{1}{|l|}{\textit{Descrizione}} & L'\textit{Analisi dei Requisiti} è parte fondamentale per la realizzazione del prodotto finale, per questo è necessario che sia completa ed esaustiva. \\ \hline
    \multicolumn{1}{|l|}{\textit{Mitigazione}} & L'\textit{Analisi dei Requisiti} è stata realizzata approfondendo il più possibile i casi d'uso e i requisiti anche grazie ad un confronto diretto con il proponente del progetto in modo che fosse ben chiaro come dovrà essere il prodotto finale. Questo ha permesso al gruppo di effettuare, fino a questo momento, un'analisi soddisfacente. Sono state quindi rispettate le precauzioni. \\ \hline
    \end{tabular}
\end{table}
