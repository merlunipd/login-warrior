\chapter{Consuntivo}

Nel consuntivo, vengono riprese le tabelle del preventivo. Al posto dei valori inseriti nel preventivo, si inserisce:
\begin{itemize}
    \item Valore effettivo consuntivato;
    \item Se il valore precedente è diverso da quello del preventivo, due parentesi tonde con all'interno
        la differenza tra valore di consuntivo e valore di preventivo.
\end{itemize}

\section{Verso la RTB}

\subsection{Primo periodo}

\subsubsection{Consuntivo orario}
\begin{table}[!ht]
    \centering
    \begin{tabular}{|l|c|c|c|c|c|c|c|}
    \hline
    \textbf{Membro} & \multicolumn{1}{c|}{\textbf{RE}} & \multicolumn{1}{c|}{\textbf{AM}} & \multicolumn{1}{c|}{\textbf{AN}} & \multicolumn{1}{c|}{\textbf{PT}} & \multicolumn{1}{c|}{\textbf{PR}} & \multicolumn{1}{c|}{\textbf{VE}} & \multicolumn{1}{c|}{\textbf{Totale ore persona}} \\ \hline
    \textit{Marco Mazzucato}  & 3 (+1) & 3          & - & - & - & 1         & 7 (+1)   \\ \hline
    \textit{Marco Mamprin}    & -      & 2 (-1)     & - & - & - & 1.5 (+0.5)& 3.5 (-0.5) \\ \hline
    \textit{Marko Vukovic}    & 2      & 3          & - & - & - & 1         & 6        \\ \hline
    \textit{Mattia Zanellato} & -      & 1.5 (-1.5) & - & - & - & 2.5 (+1.5)& 4        \\ \hline
    \textit{Emanuele Pase}    & -      & 2.5 (-0.5) & - & - & - & 1         & 3.5 (-0.5) \\ \hline
    \textit{Riccardo Contin}  & -      & 3          & - & - & - & 1         & 4        \\ \hline
    \textit{Lorenzo Onelia}   & -      & 2.5 (-0.5) & - & - & - & 2 (+1)    & 4.5 (+0.5) \\ \hline
    \textbf{Totale ore ruolo} & 5 (+1) & 17.5 (-3.5)& - & - & - & 10 (+3)   & 32.5 (+0.5)\\ \hline
    \end{tabular}
    \caption{Distribuzione delle ore per la prima milestone}
\end{table}

\begin{table}[!ht]
    \centering
    \begin{tabular}{|l|c|c|c|c|c|c|c|}
    \hline
    \textbf{Membro} & \multicolumn{1}{l|}{\textbf{RE}} & \multicolumn{1}{l|}{\textbf{AM}} & \multicolumn{1}{l|}{\textbf{AN}} & \multicolumn{1}{l|}{\textbf{PT}} & \multicolumn{1}{l|}{\textbf{PR}} & \multicolumn{1}{l|}{\textbf{VE}} & \multicolumn{1}{l|}{\textbf{Totale ore persona}} \\ \hline
    \textit{Marco Mazzucato}  & 6  & 5   & 5  & 23  & 24 & 25   & 88   \\ \hline
    \textit{Marco Mamprin}    & 9  & 6   & 5  & 23  & 24 & 24.5 & 91.5 \\ \hline
    \textit{Marko Vukovic}    & 7  & 5   & 5  & 23  & 24 & 25   & 89   \\ \hline
    \textit{Mattia Zanellato} & 9  & 6.5 & 5  & 23  & 24 & 23.5 & 91   \\ \hline
    \textit{Emanuele Pase}    & 9  & 5.5 & 5  & 23  & 24 & 25   & 91.5 \\ \hline
    \textit{Riccardo Contin}  & 9  & 5   & 5  & 23  & 24 & 25   & 91   \\ \hline
    \textit{Lorenzo Onelia}   & 9  & 5.5 & 5  & 23  & 24 & 24   & 90.5 \\ \hline
    \textbf{Totale ore ruolo} & 58 & 38.5& 35 & 161 & 168& 172  & 632.5\\ \hline
    \end{tabular}
    \caption{Ore rimaste dopo la prima milestone}
\end{table}

\subsubsection{Consuntivo economico}

\begin{table}[!ht]
    \centering
    \begin{tabular}{|l|c|c|}
    \hline
    \textbf{Ruolo} & \multicolumn{1}{l|}{\textbf{Ore}} & \multicolumn{1}{l|}{\textbf{Costo (€)}} \\ \hline
    \textit{Responsabile}      & 5 (+1)    & 150 (+30) \\ \hline
    \textit{Amministratore}    & 17.5 (-3.5) & 350 (-70) \\ \hline
    \textit{Analista}          & -         & -         \\ \hline
    \textit{Progettista}       & -         & -         \\ \hline
    \textit{Programmatore}     & -         & -         \\ \hline
    \textit{Verificatore}      & 10 (+3)    & 150 (+45) \\ \hline
    \textbf{Totale Preventivo} & 32        & 645       \\ \hline
    \textbf{Totale Consuntivo} & 32.5      & 650       \\ \hline
    \textbf{Differenza}        & +0.5      & +5       \\ \hline
    \end{tabular}
    \caption{Consuntivo dei costi per la prima milestone}
\end{table}

\subsubsection{Conclusioni}
Dal consuntivo si può dedurre che il gruppo è stato quasi in linea con il preventivo di periodo.
L'unica differenza è stata che sono servite meno ore di \textit{Amministratore} e più
ore di \textit{Responsabile} e \textit{Verificatore}. Questo ha portato a una spesa maggiore di 5€ per un
complessivo di 650€ a fronte dei 645€ previsti.
In conclusione il budget rimanente è di \num{12510}€.



\subsection{Secondo periodo}

\subsubsection{Consuntivo orario}
\begin{table}[H]
    \centering
    \begin{tabular}{|l|c|c|c|c|c|c|c|}
    \hline
    \textbf{Membro} & \multicolumn{1}{c|}{\textbf{RE}} & \multicolumn{1}{c|}{\textbf{AM}} & \multicolumn{1}{c|}{\textbf{AN}} & \multicolumn{1}{c|}{\textbf{PT}} & \multicolumn{1}{c|}{\textbf{PR}} & \multicolumn{1}{c|}{\textbf{VE}} & \multicolumn{1}{c|}{\textbf{Totale ore persona}} \\ \hline
    \textit{Marco Mazzucato}  & -      & 2            & 2          & - & - & 1 (-1)        & 5 (-1)         \\ \hline
    \textit{Marco Mamprin}    & -      & 2            & 0.5 (-0.5)   & - & - & 1 (-1)        & 3.5 (-1.5)       \\ \hline
    \textit{Marko Vukovic}    & -      & 1.5          & 2 (-1)     & - & - & 3             & 6.5 (-1)       \\ \hline
    \textit{Mattia Zanellato} & -      & -            & 3          & - & - & 2.5 (-0.5)      & 5.5 (-0.5)       \\ \hline
    \textit{Emanuele Pase}    & -      & -            & 3          & - & - & 2.5 (-0.5)      & 5.5 (-0.5)       \\ \hline
    \textit{Riccardo Contin}  & 4      & -            & 2 (-1)     & - & - & 1.5 (+0.5)      & 7.5 (-0.5)       \\ \hline
    \textit{Lorenzo Onelia}   & -      & 0 (-2)       & 1 (-1)     & - & - & 3.5 (+0.5)      & 4.5 (-2.5)       \\ \hline
    \textbf{Totale ore ruolo} & 4      & 5.5 (-2)     & 13.5 (-3.5)  & - & - & 15 (-2)       & 38 (-7.5)    \\ \hline
    \end{tabular}
    \caption{Distribuzione delle ore per la seconda milestone}
\end{table}

\begin{table}[H]
    \centering
    \begin{tabular}{|l|c|c|c|c|c|c|c|}
    \hline
    \textbf{Membro} & \multicolumn{1}{l|}{\textbf{RE}} & \multicolumn{1}{l|}{\textbf{AM}} & \multicolumn{1}{l|}{\textbf{AN}} & \multicolumn{1}{l|}{\textbf{PT}} & \multicolumn{1}{l|}{\textbf{PR}} & \multicolumn{1}{l|}{\textbf{VE}} & \multicolumn{1}{l|}{\textbf{Totale ore persona}} \\ \hline
    \textit{Marco Mazzucato}  & 6  & 3    & 3    & 23  & 24 & 24     & 83     \\ \hline
    \textit{Marco Mamprin}    & 9  & 4    & 4.5  & 23  & 24 & 23.5   & 88     \\ \hline
    \textit{Marko Vukovic}    & 7  & 3.5  & 3    & 23  & 24 & 22     & 82.5   \\ \hline
    \textit{Mattia Zanellato} & 9  & 6.5  & 2    & 23  & 24 & 21     & 85.5   \\ \hline
    \textit{Emanuele Pase}    & 9  & 5.5  & 2    & 23  & 24 & 22.5   & 86     \\ \hline
    \textit{Riccardo Contin}  & 5  & 5    & 3    & 23  & 24 & 23.5   & 83.5   \\ \hline
    \textit{Lorenzo Onelia}   & 9  & 5.5  & 4    & 23  & 24 & 20.5   & 86     \\ \hline
    \textbf{Totale ore ruolo} & 54 & 33   & 21.5 & 161 & 168& 157    & 594.5  \\ \hline
    \end{tabular}
    \caption{Ore rimaste dopo la seconda milestone}
\end{table}

\subsubsection{Consuntivo economico}

\begin{table}[H]
    \centering
    \begin{tabular}{|l|c|c|}
    \hline
    \textbf{Ruolo} & \multicolumn{1}{l|}{\textbf{Ore}} & \multicolumn{1}{l|}{\textbf{Costo (€)}} \\ \hline
    \textit{Responsabile}      & 4         & 120         \\ \hline
    \textit{Amministratore}    & 5.5 (-2)  & 110 (-40)   \\ \hline
    \textit{Analista}          & 13.5(-3.5)  & 337.5 (-87.5) \\ \hline
    \textit{Progettista}       & -         & -           \\ \hline
    \textit{Programmatore}     & -         & -           \\ \hline
    \textit{Verificatore}      & 15 (-2)   & 225 (-30)   \\ \hline
    \textbf{Totale Preventivo} & 45.5      & 950         \\ \hline
    \textbf{Totale Consuntivo} & 38        & 792.5       \\ \hline
    \textbf{Differenza}        & -7.5      & -157.5      \\ \hline
    \end{tabular}
    \caption{Consuntivo dei costi per la seconda milestone}
\end{table}

\subsubsection{Conclusioni}
Come si può notare dal consuntivo il gruppo non è riuscito a rimanere in linea con quanto preventivato.
\\Le differenze dal preventivo riguardano i ruoli di \textit{Amministratore} (-2 ore svolte), \textit{Analista} (-3.5 ore svolte) e \textit{Verificatore} (-2 ore svolte).
Questo ha portato ad una riduzione della spesa totale preventivata di 157.5€ e una riduzione delle ore produttive pari a 7.5.
\\Tra le cause della discrepanza tra quanto preventivato e quanto consuntivato possiamo individuare:
    \begin{itemize}
        \item L'indisponibilità del proponente ad un incontro nel periodo festivo;
        \item La presenza di festività che hanno rallentato l'avanzamento dei lavori più di quanto previsto;
        \item L'errata stima di disponibilità oraria di alcuni membri del gruppo.
    \end{itemize}
Per migliorare la prescisione dei prossimi preventivi il gruppo ha deciso di preventivare solamente ore che con molta probabilità verranno svolte,
preferendo comunque aggiungere ore al consuntivo invece che toglierle.
\\Il budget rimanente è di \num{11717,5}€.


\subsection{Terzo periodo}

\subsubsection{Consuntivo orario}
\begin{table}[H]
    \centering
    \begin{tabular}{|l|c|c|c|c|c|c|c|}
    \hline
    \textbf{Membro} & \multicolumn{1}{c|}{\textbf{RE}} & \multicolumn{1}{c|}{\textbf{AM}} & \multicolumn{1}{c|}{\textbf{AN}} & \multicolumn{1}{c|}{\textbf{PT}} & \multicolumn{1}{c|}{\textbf{PR}} & \multicolumn{1}{c|}{\textbf{VE}} & \multicolumn{1}{c|}{\textbf{Totale ore persona}} \\ \hline
    \textit{Marco Mazzucato}  & -    & 0.5 (-0.5)     & 1 (-0.5)  & 0.5 (-1.5)   & - & 1 (-1)        & 3 (-3.5)       \\ \hline
    \textit{Marco Mamprin}    & -    & 1            & 0.5 (-1.5)    & 0.5 (-1.5)   & - & 2             & 4 (-3)           \\ \hline
    \textit{Marko Vukovic}    & -    & 1.5          & 2           & 2          & - & 2             & 7.5              \\ \hline
    \textit{Mattia Zanellato} & -    & 0.5 (-0.5)     & 0.5 (-1.5)    & 1.5 (-0.5)   & - & 1.5 (-0.5)      & 4 (-3)           \\ \hline
    \textit{Emanuele Pase}    & 4    & 1            & 0.5         & 2          & - & 0.5 (-0.5)      & 8 (-0.5)       \\ \hline
    \textit{Riccardo Contin}  & -    & 1            & 1 (-0.5)  & 2          & - & 2 (-0.5)    & 6 (-1)           \\ \hline
    \textit{Lorenzo Onelia}   & -    & 1            & - (-1.5)  & 1 (-1)     & - & 1 (-1)        & 3 (-3.5)       \\ \hline
    \textbf{Totale ore ruolo} & 4    & 6.5 (-1)     & 5.5 (-5.5)   & 9.5 (-4.5)  & - & 10 (-3.5)   & 35.5 (-14.5)       \\ \hline
    \end{tabular}
  \caption{Distribuzione delle ore per la terza milestone}
\end{table}

\begin{table}[H]
    \centering
    \begin{tabular}{|l|c|c|c|c|c|c|c|}
    \hline
    \textbf{Membro} & \multicolumn{1}{l|}{\textbf{RE}} & \multicolumn{1}{l|}{\textbf{AM}} & \multicolumn{1}{l|}{\textbf{AN}} & \multicolumn{1}{l|}{\textbf{PT}} & \multicolumn{1}{l|}{\textbf{PR}} & \multicolumn{1}{l|}{\textbf{VE}} & \multicolumn{1}{l|}{\textbf{Totale ore persona}} \\ \hline
    \textit{Marco Mazzucato}  & 6  & 2.5  & 2    & 22.5  & 24  & 23     & 80     \\ \hline
    \textit{Marco Mamprin}    & 9  & 3    & 4    & 22.3  & 24  & 21.5   & 84     \\ \hline
    \textit{Marko Vukovic}    & 7  & 2    & 1    & 21    & 24  & 20     & 75     \\ \hline
    \textit{Mattia Zanellato} & 9  & 6    & 1.5  & 21.5  & 24  & 19.5   & 81.5   \\ \hline
    \textit{Emanuele Pase}    & 5  & 4.5  & 1.5  & 21    & 24  & 22     & 78     \\ \hline
    \textit{Riccardo Contin}  & 5  & 4    & 2    & 21    & 24  & 21.5   & 77.5   \\ \hline
    \textit{Lorenzo Onelia}   & 9  & 4.5  & 4    & 22    & 24  & 19.5   & 83     \\ \hline
    \textbf{Totale ore ruolo} & 50 & 26.5 & 16   & 151.5 & 168 & 147    & 559    \\ \hline
    \end{tabular}
    \caption{Ore rimaste dopo la terza milestone}
\end{table}

\subsubsection{Consuntivo economico}

\begin{table}[H]
    \centering
    \begin{tabular}{|l|c|c|}
    \hline
    \textbf{Ruolo} & \multicolumn{1}{l|}{\textbf{Ore}} & \multicolumn{1}{l|}{\textbf{Costo (€)}} \\ \hline
    \textit{Responsabile}      & 4           & 120            \\ \hline
    \textit{Amministratore}    & 6.5 (-1)    & 130 (-20)      \\ \hline
    \textit{Analista}          & 5.5 (-5.5)   & 137.5 (-137.5)   \\ \hline
    \textit{Progettista}       & 9.5 (-4.5)   & 237.5 (-112.5)   \\ \hline
    \textit{Programmatore}     & -           & -              \\ \hline
    \textit{Verificatore}      & 10 (-3.5) & 150 (-52.5)  \\ \hline
    \textbf{Totale Preventivo} & 50          & 1097.5         \\ \hline
    \textbf{Totale Consuntivo} & 35.5        & 775            \\ \hline
    \textbf{Differenza}        & -14.5       & -322.5         \\ \hline
    \end{tabular}
    \caption{Consuntivo dei costi per la terza milestone}
\end{table}

\subsubsection{Conclusioni}
Il consuntivo può chiaramente evidenziare che il gruppo non è riuscito a rimanere in linea con quanto preventivato.
\\I ruoli in cui si possono notare differenze dal preventivo sono i seguenti: \textit{Amministratore} (-1 ora svolta), \textit{Analista} (-5.5 ore svolte), \textit{Progettista} (-4.5 ore svolte) e \textit{Verificatore} (-3.5 ore svolte).
In seguito alla diversità tra preventivo e consuntivo si può notare una riduzione della spesa totale preventivata di 322.5€ e una riduzione delle ore produttive pari a 14.5.
\\Tra le principali cause di questa disuguaglianza tra quanto preventivato e quanto consuntivato possiamo individuare:
    \begin{itemize}
        \item La presenza della sessione d'esami che ha occupato più tempo del previsto per alcuni membri del gruppo;
        \item L'errata stima di disponibilità oraria di alcuni membri del gruppo.
    \end{itemize}
Dato il ripetuto errore nella stime di ore disponibili il gruppo ha deciso che ogni membro dovrà ritagliarsi una porzione di tempo per pensare più nello specifico alla propria disponibilità oraria in modo da evitare di commettere errori simili.
\\Il budget rimanente è di \num{10942.5}€.


\section{Verso la PB}

\section{Verso la CA}
