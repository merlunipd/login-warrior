\chapter{Analisi dei rischi}

In un progetto che prevede la realizzazione, da parte di un insieme di persone, di un prodotto concreto 
da consegnare a un proponente$_G$ è inevitabile che possano verificarsi dei problemi più o meno gravi che
provochino rallentamenti. Per tentare di evitare che accada è necessario fare un'attenta analisi dei
rischi. La nostra aspettativa è quella di riuscire a effettuare delle scelte che ci permettano di
incontrare meno problemi possibili e nel caso se ne verifichi qualcuno avere già pronta una soluzione.

Per organizzare l'analisi ciò che vogliamo evidenziare per ogni rischio è:
\begin{itemize}
    \item Rischio;
    \item Descrizione;
    \item Probabilità di occorrenza;
    \item Grado di pericolosità;
    \item Precauzione;
    \item Piano di contingenza$_G$.
\end{itemize}

\newpage
\section{Rischi legati alle persone}


\begin{table}[H]
    \renewcommand\arraystretch{1.35}
    \centering
    \begin{tabular}{|p{3cm}|p{11cm}|}
    \hline
    \rowcolor[HTML]{036400}
    \multicolumn{2}{|c|}{\textcolor{white}{\textbf{Disponibilità}}} \\ \hline
    \rowcolor[HTML]{EFEFEF}\multicolumn{1}{|l|}{\textit{Descrizione}} & \begin{tabular}[c]{@{}l@{}}Ogni membro dovrà affrontare questo progetto sapendo\\ di avere altri impegni universitari e personali.\\ Questo può provocare momenti di inattività o scarsa\\ partecipazione.\end{tabular} \\ \hline
    \rowcolor[HTML]{C0C0C0}\multicolumn{1}{|l|}{\textit{\begin{tabular}[c]{@{}l@{}}Probabilità di\\ occorenza\end{tabular}}} & Alta. \\ \hline
    \rowcolor[HTML]{EFEFEF}\multicolumn{1}{|l|}{\textit{\begin{tabular}[c]{@{}l@{}}Grado di\\ pericolosità\end{tabular}}} & Media. \\ \hline
    \rowcolor[HTML]{C0C0C0}\multicolumn{1}{|l|}{\textit{Precauzioni}} & \begin{tabular}[c]{@{}l@{}}Ogni membro dovrà essere in grado di organizzarsi al\\ meglio in modo da ritagliarsi il tempo necessario alla\\ realizzazione del progetto. Nel caso in cui un membro\\ non abbia nessuna possibilità di operare per un \\ determinato periodo deve avvisare il gruppo.\end{tabular} \\ \hline
    \rowcolor[HTML]{EFEFEF}\multicolumn{1}{|l|}{\textit{\begin{tabular}[c]{@{}l@{}}Piano di\\ contingenza\end{tabular}}} & \begin{tabular}[c]{@{}l@{}}Se la mancanza di uno o più membri sta provocando \\ ritardo, il \textit{Responsabile} deve ripianificare la suddivisione\\ del lavoro.\end{tabular} \\ \hline
    \end{tabular}
\end{table}

\begin{table}[H]
    \renewcommand\arraystretch{1.35}
    \centering
    \begin{tabular}{|p{3cm}|p{11cm}|}
    \hline
    \rowcolor[HTML]{036400}
    \multicolumn{2}{|c|}{\textcolor{white}{\textbf{Problemi interpersonali}}} \\ \hline
    \rowcolor[HTML]{EFEFEF}\multicolumn{1}{|l|}{\textit{Descrizione}} & \begin{tabular}[c]{@{}l@{}}I gruppi sono stati formati casualmente e per questo\\ è possibile che i membri non si conoscano.\\ C'è il rischio che alcuni membri non vadano \\ molto d'accordo o che non ci sia la massima \\ collaborazione.\end{tabular} \\ \hline
    \rowcolor[HTML]{C0C0C0}\multicolumn{1}{|l|}{\textit{\begin{tabular}[c]{@{}l@{}}Probabilità di\\ occorenza\end{tabular}}} & Bassa. \\ \hline
    \rowcolor[HTML]{EFEFEF}\multicolumn{1}{|l|}{\textit{\begin{tabular}[c]{@{}l@{}}Grado di\\ pericolosità\end{tabular}}} & Alta. \\ \hline
    \rowcolor[HTML]{C0C0C0}\multicolumn{1}{|l|}{\textit{Precauzioni}} & \begin{tabular}[c]{@{}l@{}}Prima di iniziare a lavorare i membri devono imparare a \\ conoscersi ed evitare contrasti ma piuttosto discuterne \\ positivamente.\end{tabular} \\ \hline
    \rowcolor[HTML]{EFEFEF}\multicolumn{1}{|l|}{\textit{\begin{tabular}[c]{@{}l@{}}Piano di\\ contingenza\end{tabular}}} & \begin{tabular}[c]{@{}l@{}}Nel caso si verifichino scontri o ci sia poca collaborazione, \\ il \textit{Responsabile} deve bloccare il progetto e cercare, \\ con la massima partecipazione di tutti, di risolvere il \\ problema.\end{tabular} \\ \hline
    \end{tabular}
\end{table}

\begin{table}[H]
    \renewcommand\arraystretch{1.35}
    \centering
    \begin{tabular}{|p{3cm}|p{11cm}|}
    \hline
    \rowcolor[HTML]{036400}
    \multicolumn{2}{|c|}{\textcolor{white}{\textbf{Mancanza di esperienza personale}}} \\ \hline
    \rowcolor[HTML]{EFEFEF}\multicolumn{1}{|l|}{\textit{Descrizione}} & \begin{tabular}[c]{@{}l@{}}Data la poca esperienza, ogni membro potrebbe trovarsi\\ in difficoltà durante il progetto.\end{tabular} \\ \hline
    \rowcolor[HTML]{C0C0C0}\multicolumn{1}{|l|}{\textit{\begin{tabular}[c]{@{}l@{}}Probabilità di\\ occorenza\end{tabular}}} & Alta. \\ \hline
    \rowcolor[HTML]{EFEFEF}\multicolumn{1}{|l|}{\textit{\begin{tabular}[c]{@{}l@{}}Grado di\\ pericolosità\end{tabular}}} & Alta. \\ \hline
    \rowcolor[HTML]{C0C0C0}\multicolumn{1}{|l|}{\textit{Precauzioni}} & \begin{tabular}[c]{@{}l@{}}Il gruppo dovrà supportarsi a vicenda cercando di aiutare\\ un membro in difficoltà.\end{tabular} \\ \hline
    \rowcolor[HTML]{EFEFEF}\multicolumn{1}{|l|}{\textit{\begin{tabular}[c]{@{}l@{}}Piano di\\ contingenza\end{tabular}}} & \begin{tabular}[c]{@{}l@{}}Se un membro trova una difficoltà prima di tutto deve\\ tentare di affrontarla, solo successivamente deve chiedere\\ il supporto del \textit{Responsabile} che si preoccuperà di\\ stabilizzare la situazione.\end{tabular} \\ \hline
    \end{tabular}
\end{table}

\section{Rischi legati all'organizzazione}

\begin{table}[H]
    \renewcommand\arraystretch{1.35}
    \centering
    \begin{tabular}{|p{3cm}|p{11cm}|}
    \hline
    \rowcolor[HTML]{036400}
    \multicolumn{2}{|c|}{\textcolor{white}{\textbf{Scarsa Pianificazione}}} \\ \hline
    \rowcolor[HTML]{EFEFEF}\multicolumn{1}{|l|}{\textit{Descrizione}} & \begin{tabular}[c]{@{}l@{}}Pianificare un intero progetto, individuando le attività e\\ suddividendo i compiti, non è un aspetto facile e questo\\ può provocare ritardi e spreco di risorse.\end{tabular} \\ \hline
    \rowcolor[HTML]{C0C0C0}\multicolumn{1}{|l|}{\textit{\begin{tabular}[c]{@{}l@{}}Probabilità di\\ occorenza\end{tabular}}} & Alta. \\ \hline
    \rowcolor[HTML]{EFEFEF}\multicolumn{1}{|l|}{\textit{\begin{tabular}[c]{@{}l@{}}Grado di\\ pericolosità\end{tabular}}} & Alta. \\ \hline
    \rowcolor[HTML]{C0C0C0}\multicolumn{1}{|l|}{\textit{Precauzioni}} & \begin{tabular}[c]{@{}l@{}}All'inizio la pianificazione deve essere un po'\\ pessimistica ponendo milestone ravvicinate con obiettivi\\ chiari in modo che sia più semplice effettuare correzioni.\end{tabular} \\ \hline
    \rowcolor[HTML]{EFEFEF}\multicolumn{1}{|l|}{\textit{\begin{tabular}[c]{@{}l@{}}Piano di\\ contingenza\end{tabular}}} & \begin{tabular}[c]{@{}l@{}}Se dopo un consuntivo ci si rende conto che la distanza \\ dal preventivo è troppo ampia è necessario ripianificare\\ i periodi successivi prima di fare qualsiasi altra attività.\end{tabular} \\ \hline
    \end{tabular}
\end{table}

\newpage
\section{Rischi legati alle tecnologie e agli strumenti}

\begin{table}[H]
    \renewcommand\arraystretch{1.35}
    \centering
    \begin{tabular}{|p{3cm}|p{11cm}|}
    \hline
    \rowcolor[HTML]{036400}
    \multicolumn{2}{|c|}{\textcolor{white}{\textbf{Strumenti sconosciuti}}} \\ \hline
    \rowcolor[HTML]{EFEFEF}\multicolumn{1}{|l|}{\textit{Descrizione}} & \begin{tabular}[c]{@{}l@{}}A supporto di un buon progetto ci sono degli ottimi\\ strumenti che però non sono immediati da capire e da\\ riuscire ad utilizzare.\end{tabular} \\ \hline
    \rowcolor[HTML]{C0C0C0}\multicolumn{1}{|l|}{\textit{\begin{tabular}[c]{@{}l@{}}Probabilità di\\ occorenza\end{tabular}}} & Media. \\ \hline
    \rowcolor[HTML]{EFEFEF}\multicolumn{1}{|l|}{\textit{\begin{tabular}[c]{@{}l@{}}Grado di\\ pericolosità\end{tabular}}} & Media. \\ \hline
    \rowcolor[HTML]{C0C0C0}\multicolumn{1}{|l|}{\textit{Precauzioni}} & \begin{tabular}[c]{@{}l@{}}Prima di utilizzarne uno, si fa una visione di gruppo\\ dell'utilizzo. Ogni membro poi si preoccuperà di \\ esercitarsi e comprendere tutti gli aspetti utili che lo \\ strumento può offrire senza focalizzarsi troppo su ciò \\ che può risultare inutile.\end{tabular} \\ \hline
    \rowcolor[HTML]{EFEFEF}\multicolumn{1}{|l|}{\textit{\begin{tabular}[c]{@{}l@{}}Piano di\\ contingenza\end{tabular}}} & \begin{tabular}[c]{@{}l@{}}Se l'utilizzo di uno strumento causa ritardi e non è\\ funzionale al progetto, si cerca un alternativa o si\\ valuta di non utilizzarlo direttamente.\end{tabular} \\ \hline
    \end{tabular}
\end{table}

\begin{table}[H]
    \renewcommand\arraystretch{1.35}
    \centering
    \begin{tabular}{|p{3cm}|p{11cm}|}
    \hline
    \rowcolor[HTML]{036400}
    \multicolumn{2}{|c|}{\textcolor{white}{\textbf{Tecnologie sconosciute}}} \\ \hline
    \rowcolor[HTML]{EFEFEF}\multicolumn{1}{|l|}{\textit{Descrizione}} & \begin{tabular}[c]{@{}l@{}}Per la codifica del prodotto software è importante\\ individuare quali siano le tecnologie presenti nel\\ mercato più adatte a ciò che si vuole realizzare. Bisogna\\ però tenere in considerazione che queste possono\\ essere completamente sconosciute ad uno o più membri.\end{tabular} \\ \hline
    \rowcolor[HTML]{C0C0C0}\multicolumn{1}{|l|}{\textit{\begin{tabular}[c]{@{}l@{}}Probabilità di\\ occorenza\end{tabular}}} & Alta. \\ \hline
    \rowcolor[HTML]{EFEFEF}\multicolumn{1}{|l|}{\textit{\begin{tabular}[c]{@{}l@{}}Grado di\\ pericolosità\end{tabular}}} & Alta. \\ \hline
    \rowcolor[HTML]{C0C0C0}\multicolumn{1}{|l|}{\textit{Precauzioni}} & \begin{tabular}[c]{@{}l@{}}Quando si discute di una tecnologia da utilizzare, ogni\\ membro deve esprimere se la conosce e il suo livello.\\ Questo permetterà di capire se la scelta è giusta, ma\\ suggerirà anche la migliore suddivisione del lavoro.\end{tabular} \\ \hline
    \rowcolor[HTML]{EFEFEF}\multicolumn{1}{|l|}{\textit{\begin{tabular}[c]{@{}l@{}}Piano di\\ contingenza\end{tabular}}} & \begin{tabular}[c]{@{}l@{}}Se l'utilizzo di una tecnologia causa ritardi e non è\\ funzionale al progetto, si cerca un alternativa o si punta\\ maggiormente sulla collaborazione per riuscire a trovare\\ una soluzione mettendo insieme le conoscenze.\end{tabular} \\ \hline
    \end{tabular}
\end{table}

\newpage
\section{Rischi legati ai requisiti}

\begin{table}[H]
    \renewcommand\arraystretch{1.35}
    \centering
    \begin{tabular}{|p{3cm}|p{11cm}|}
    \hline
    \rowcolor[HTML]{036400}
    \multicolumn{2}{|c|}{\textcolor{white}{\textbf{\textit{Analisi dei requisiti} incompleta}}} \\ \hline
    \rowcolor[HTML]{EFEFEF}\multicolumn{1}{|l|}{\textit{Descrizione}} & \begin{tabular}[c]{@{}l@{}}L'\textit{Analisi dei requisiti} è un documento molto importante\\ per la buona realizzazione del prodotto. Se questo però è\\ incompleto o mal fatto, allora sicuramente il risultato\\ finale non sarà del tutto soddisfacente o ancora peggio\\ si incontreranno delle difficoltà.\end{tabular} \\ \hline
    \rowcolor[HTML]{C0C0C0}\multicolumn{1}{|l|}{\textit{\begin{tabular}[c]{@{}l@{}}Probabilità di\\ occorenza\end{tabular}}} & Media. \\ \hline
    \rowcolor[HTML]{EFEFEF}\multicolumn{1}{|l|}{\textit{\begin{tabular}[c]{@{}l@{}}Grado di\\ pericolosità\end{tabular}}} & Alta. \\ \hline
    \rowcolor[HTML]{C0C0C0}\multicolumn{1}{|l|}{\textit{Precauzioni}} & \begin{tabular}[c]{@{}l@{}}Approfondire bene tutti i casi d'uso e i vari requisiti\\ mantenendo una conversazione aperta con il proponente.\end{tabular} \\ \hline
    \rowcolor[HTML]{EFEFEF}\multicolumn{1}{|l|}{\textit{\begin{tabular}[c]{@{}l@{}}Piano di\\ contingenza\end{tabular}}} & Discutere con il proponente. \\ \hline
    \end{tabular}
\end{table}


