\chapter{Introduzione}

\section{Scopo del documento}

Il \textit{Piano di Progetto V2.0.0} è un documento di fondamentale importanza per riuscire a lavorare nel migliore dei
modi. La sua struttura è:
\begin{itemize}
    \item \textbf{Analisi dei rischi: } permette di indicare i possibili rischi, la probabilità che essi si verifichino e la loro gravità;
    \item \textbf{Pianificazione: } permette di pianificare le milestone$_G$;
    \item \textbf{Preventivo}$_G$ : permette di indicare le ore e i costi che si intende impiegare in ogni periodo pianificato;
    \item \textbf{Consuntivo}$_G$ : permette di analizzare il reale svolgimento dei periodi passati rispetto a com'erano stati preventivati;
    \item \textbf{Mitigazione dei rischi: } permette di analizzare i rischi che si sono effettivamente verificati.
\end{itemize}

\section{Glossario}
Al fine di evitare incomprensioni relative alla terminologia usata all'interno del documento, viene fornito un Glossario nel file \textit{Glossario V2.0.0} in grado di dare una definizione precisa per ogni vocabolo potenzialmente ambiguo. Tali termini verranno evidenziati all'interno del documento con una G in pedice.

\section{Riferimenti}
\subsection{Riferimenti normativi}
\begin{itemize}
  \item \textit{Norme di Progetto V2.0.0}
  \item Capitolato d'appalto C5 - Zucchetti S.p.A.: Login Warrior \\
  \url{https://www.math.unipd.it/~tullio/IS-1/2021/Progetto/C5.pdf}
\end{itemize}

\subsection{Riferimenti informativi}
\begin{itemize}
  \item Slide T5 - Corso di Ingegneria del Software - Il ciclo di vita del SW \\
  \url{https://www.math.unipd.it/~tullio/IS-1/2021/Dispense/T05.pdf}
  \item Slide T6 - Corso di Ingegneria del Software - Gestione di progetto \\
  \url{https://www.math.unipd.it/~tullio/IS-1/2021/Dispense/T06.pdf}
\end{itemize}
