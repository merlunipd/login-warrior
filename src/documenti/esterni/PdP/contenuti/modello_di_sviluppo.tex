\chapter{Modello di sviluppo}

\section{Modello agile}
Il gruppo \textit{MERL} ha deciso di ispirarsi ai modelli agili per lo sviluppo del progetto. Questo prevede rilasci multipli e successivi, ciascuno dei quali è in grado di realizzare un incremento di funzionalità.
\\Adottando questo modello risulta necessario individuare e classificare i requisiti$_G$ in modo da poter dare un ordine allo sviluppo che permetterà di ottenere dopo ogni incremento un prodotto, seppur incompleto, stabile e funzionante. Per mantenere questo è fondamentale che i primi incrementi vadano a soddisfare i requisiti più importanti in modo tale da renderli fin da subito stabili all'interno del prodotto. Solo successivamente verranno integrati i requisiti di minor importanza, che avranno dunque più tempo per stabilizzarsi nel prodotto.
\\L'utilizzo dei modelli agili porta in particolare i seguenti vantaggi:
\begin{itemize}
  \item Viene data priorità allo sviluppo delle funzionalità primarie, questo permette una costante verifica$_G$ anche da parte del proponente$_G$ delle principali funzionalità;
  \item Avendo un prodotto funzionante sarà possibile ottenere, dopo ogni incremento, un riscontro da parte del proponente che potrà quindi valutare il funzionamento del prodotto fino a quel momento;
  \item Gli errori saranno facilmente individuabili dato che ogni incremento prevede una fase di verifica finale, questo permette un minor dispendio di risorse per l'individuazione e la risoluzione di tali errori;
  \item La verifica e i test fatti saranno più semplici in quanto saranno volutamente mirati alle modifiche effettuate durante uno specifico incremento;
  \item Permette di rispondere ai cambiamenti in modo molto efficiente, permettendo di rimanere in linea con le aspettative del proponente.
\end{itemize}
