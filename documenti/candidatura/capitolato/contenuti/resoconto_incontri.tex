\section{Resoconto degli incontri}
L’incontro con il proponente concordato è stato svolto in data 3 Novembre 2021 sulla piattaforma Zoom. Il colloquio è stato di natura principalmente conoscitiva e successivamente sono stati presentati alcuni dubbi sia di tipo organizzativo sia di tipo tecnico riguardo al progetto. Inoltre il proponente ha voluto esprimere in modo più chiaro gli obiettivi concludendo con alcuni consigli generali.

\begin{spacing}{2}
\end{spacing}

\begin{minipage}[b]{0.47\textwidth}
\centering
\textbf{Domande}
\end{minipage}
\hfill
\begin{minipage}[b]{0.47\textwidth}
\centering
\textbf{Risposte}
\end{minipage}

\begin{spacing}{3}
\end{spacing}

\begin{minipage}[c]{0.47\textwidth}
\centering
Quanto e che tipo di supporto offre Zucchetti? Incontri e comunicazione asincrona?
\end{minipage}
\hfill
\begin{minipage}[c]{0.47\textwidth}
\centering
L’azienda è disponibile per qualsiasi tipo di dubbio e difficoltà. Ritiene non necessario più di 4/5 incontri su zoom per fare il punto della situazione durante il progetto. Per la comunicazione asincrona vanno benissimo le mail.
\end{minipage}

\begin{spacing}{3}
\end{spacing}

\begin{minipage}[c]{0.47\textwidth}
\centering
In particolare, è previsto un supporto per i requisiti opzionali di machine learning?
\end{minipage}
\hfill
\begin{minipage}[c]{0.47\textwidth}
\centering
Prima di tutto è importante portare a termine quelli che sono i requisiti obbligatori. Solo successivamente l’azienda è disposta a tenere qualche lezione sul tema richiesto.
\end{minipage}

\begin{spacing}{3}
\end{spacing}

\begin{minipage}[c]{0.47\textwidth}
\centering
Per quanto riguarda la parte visiva, il prodotto finale deve essere una vera e propria pagina web o un’applicazione locale?
\end{minipage}
\hfill
\begin{minipage}[c]{0.47\textwidth}
\centering
Pagina web client side visibile dal browser. Eventualmente si consideri la possibilità di usare Electrum per rendere la pagina un’applicazione desktop utilizzabile al di fuori del browser.
\end{minipage}

\begin{spacing}{3}
\end{spacing}

\begin{minipage}[c]{0.47\textwidth}
\centering
L’applicativo deve solo fornire una rappresentazione visiva dei dati o deve anche marcare gli utenti sospetti?
\end{minipage}
\hfill
\begin{minipage}[c]{0.47\textwidth}
\centering
I requisiti obbligatori puntano sulla rappresentazione dei dati su un grafico. Successivamente, dopo aver approfondito il tema machine learning, si può tentare di sviluppare un’identificazione delle attività sospette.
\end{minipage}

\begin{spacing}{3}
\end{spacing}

\begin{minipage}[c]{0.47\textwidth}
\centering
L’applicazione richiede un backend?
\end{minipage}
\hfill
\begin{minipage}[c]{0.47\textwidth}
\centering
Può essere completamente frontend. Eventualmente, come parte opzionale, si può considerare l’ipotesi di sviluppare un server che permetta di simulare la raccolta di login in tempo reale.
\end{minipage}

\begin{spacing}{3}
\end{spacing}

\begin{minipage}[c]{0.47\textwidth}
\centering
Per quanto riguarda il frontend sono consigliati framework? Meglio limitarsi a HTML/CSS/JS?
\end{minipage}
\hfill
\begin{minipage}[c]{0.47\textwidth}
\centering
HTML/CSS/JS vanno più che bene. A discrezione è possibile utilizzare framework consolidati o un’architettura frameworkless, ponendo però attenzione al concetto di debito tecnico.
\end{minipage}

\begin{spacing}{2}
\end{spacing}

\begin{minipage}[c]{0.47\textwidth}
\centering
Consigli riguardo la documentazione?
\end{minipage}
\hfill
\begin{minipage}[c]{0.47\textwidth}
\centering
Libertà. Considerare eventualmente l’utilizzo di Markdeep.
\end{minipage}

\begin{spacing}{3}
\end{spacing}

In seguito è stata mantenuta una corrispondenza tramite mail per ulteriori chiarimenti.
